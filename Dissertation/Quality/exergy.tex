\section{Эксергетический анализ}

Эксергетический анализ выполнен по методике [\ldots], в соответствии с которой теплотехнологическая система получения порошкообразного продукта из фильтрата спиртовой барды условно отделена от окружающей среды замкнутой контрольной поверхностью, а внутри неё с учётом теплообменных процессов выделены контрольные поверхности: I Отделение кека, фильтрация, II Выпаривание, III Эжектор, IV Конденсатор, V Насос холодильной установки, VI ТРВ, VII Распылительная сушка концентрата, VIII Парогенератор, IX Испаритель, X Теплообменник.
Красным на схеме показана пароэжекторная холодильная машина.

\begin{figure}[!htb]
\centering
\includegraphics[width=0.8\textwidth]{figures/exergy_scheme.eps}
\caption[Теплотехнологическая система получения порошкообразного продукта из фильтрата спиртовой барды]{Теплотехнологическая система получения порошкообразного продукта из фильтрата спиртовой барды:
сепараторы 1, 3; фильтры тонкой очистки 2, 4; вакуум--выпарной аппарат 8; эжектор 6; конденсатор 7; вентиль редукционный 9; сборник конденсата 14; насосы 5, 10; вентиль предохранительный 13; парогенератор 12; испаритель 14; распылительная сушилка 16; теплообменник--рекуператор 18; вентилятор 17; линии материальных потоков: 0.1 -- порошкообразный продукт; 0.8 -- кек барды; 1.0 -- отработанная вода; 1.6 -- холодная вода; 1.8 -- конденсат; 2.0 -- пар отработанный; 2.2 -- рабочий пар; 2.7 --смесь рабочего и отработанного пара; 3.0 -- отработанный сушильный агент; 3.3 -- сушильный агент; 9.1 -- исходная барда; 9.7 -- фильтрат барды; 9.8 -- сгущенный фильтрат}\label{fiq:exergy_sheme}
\end{figure}

Исходную барду подают в сепаратор 1, далее фильтрат направляют на тонкое разделение в фильтр тонкой очистки 2. 
Фильтрат подают в вакуум-выпарной аппарат 8 и далее в распылительную сушилку 16. 
Сушку проводят воздухом, который вентилятором 17 подают в корпус сушилки 16.
Разряжение в вакуум-аппарате 8 создают с помощью пароэжекторной установки производительностью 16 т/ч, включающей парогенератор 12, эжектор 6, конденсатор 7, испаритель 14, пароперегреватель 15, насос 10, ТРВ 11, работающих в замкнутом термодинамическом цикле.
Полученный в парогенераторе 12 рабочий пар разделяют на две части, одну из которых направляют в редукционный вентиль 9 и далее в греющую камеру вакуум-выпарного аппарата 5, а другую часть рабочего пара в сопло эжектора 6.
Смесь отработанного рабочего и эжектируемого паров направляют в конденсатор 7, в котором осуществляют подогрев отработанного после распылительной сушилки 16 воздуха.
Отработанный воздух из сушилки 16 для осушения отводят в теплообменник--рекуператор 18.
Затем воздух направляют сначала в конденсатор 7 для подогрева, а затем вновь в распылительную сушилку 16 с образованием замкнутого цикла.
Часть образовавшегося конденсата после теплообменника--рекуператора 18 вместе с конденсатом после греющей камеры вакуум--выпарного аппарата 8 и конденсатора 7 направляют в парогенератор 12, оснащенный предохранительным вентилем 13, для пополнения в нем уровня воды.
Другую часть конденсата насосом 5 подают в сепаратор 3 и фильтр тонкой очистки 4, для водной регенерации.


Схематическое изображение цикла работы и $T$--$S$ диаграмма холодильной установки представлены на \cref{fig:exergy_cycle,fig:exergy_t-s} соответственно. 

\begin{figure}[htb]
\centering
\includegraphics[width=0.5\textwidth]{figures/exergy_cycle.eps}
\caption{Цикл холодильной установки}\label{fig:exergy_cycle}
\end{figure}

\begin{figure}[htb]
\centering
\includegraphics[width=0.5\textwidth]{figures/exergy_t-s.eps}
\caption{$T$--$S$ диаграмма цикла холодильной установки}\label{fig:exergy_t-s}
\end{figure}

Параметры цикла холодильной установки ($h_i$ -- энтальпия, ${{ кДж}}/{{ кг}}$; $m_i$ -- массовый расход, ${{ кг}}/{{ с}}$; $P_i$ -- давление, ${ атм}$; $e_i$ -- удельная эксергия, ${{ кДж}}/{{ кг}}$; $s_i$ -- энтропия, ${{ кДж}}/{{ кг}}\cdot {\rm K}$; $T_i$ -- температура, ${\rm K}$) в точках ${ 1-12}$ приведены в~\cref{tab:exergy_sheme}.

Расчет значений эксергии $E$, ${ кВт}$, разрушения эксергии $D$, ${ кВт}$, эксергетической эффективности элемента схемы $ex$, ${ кВт}$, для точек 1--12 %, а также КПД холодильной установки ${ КПД}}_{{ max}}$ 
проводился в соответствии с уравнениями:
\begin{equation} \label{GrindEQequation54} 
E_i{ =}m_i\cdot \left(e_i{ -}e_j\right) 
\end{equation} 
где $i$, $j$ -- последовательные точки элементов схемы.
\begin{equation} \label{GrindEQequation56} 
D_i{ =}E_i{ -}E_j 
\end{equation} 
\begin{equation} \label{GrindEQequation57} 
ex_{{ i}}{ =}E_i{ /}E_j 
\end{equation} 
% \begin{equation} \label{GrindEQequation96} 
% { КПД}}_{{ max}}{ =}w_{max}\cdot \frac{h_{{ 2}}{ -}h_{{ 5}}}{h_{{ 1}}{ -}h_{{ 4}}} 
% \end{equation} 
% где $w_{max}$ -- максимальный коэффициент эжекции:
% \begin{equation} \label{GrindEQequation95} 
% w_{max}{ =}\frac{m_{{ 3}}}{m_{{ 1}}} 
% \end{equation} 

\LTXtable{\textwidth}{tabs/exergy_sheme}

Результаты расчетов эксергии сведены в \cref{tab:exergy_sheme}.
Полученный эксергетический КПД равен \(5,47\)\%, что говорит о повышении термодинамического совершенства системы при использовании пароэжекторной холодильной установки, обеспечивающей использование теплоносителей -- воздуха и горячей воды -- в режиме рециркуляции, что исключило потери эксергии в атмосферу с отходящими потоками. 
%Результаты расчетов легли в основу разработки программно--логического алгоритма управления технологией утилизации барды [\ldots].

\LTXtable{\textwidth}{tabs/exergy_results}

Таким образом, эксергетический анализ технологической линии получения порошка из фильтрата барды с использованием теплонасосной установки подтвердил практическую возможность и энергетическую эффективность предложенного решения, и позволил определить основные рабочие режимы установки.
