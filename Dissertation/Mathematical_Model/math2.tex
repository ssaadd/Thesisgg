\section{Модель подъема и выпаривания пузырька}

Предпочтение отдается компактным барботажным выпарным установкам, в которых площадь поверхности теплообмена может быть значительно изменена вариацией конструкции барботёра, в них отсутствует тенденция к постепенному снижению эффективности теплопередачи за счёт загрязнения поверхности теплообмена аппарата, что делает такие установки ненакипеобразующими.
В качестве теплоносителя используется горячий воздух, который подается в виде пузырьков непосредственно в непрерывную фазу [\ldots]. При этом выбор конструкции барботажного устройства и поиск рациональных режимных параметров процесса выпаривании фильтрата барды предлагается осуществлять
методами математического моделирования гидродинамики и теплопереноса при следующих допущениям:

-- задача рассматривается в сферической системе координат;

-- процесс испарения -- квазистатический: стационарные уравнения непрерывности определяют массовый расход испарившейся жидкости из фильтрата через поверность пузырька, а массовое уравнение непрерывности определяет скорость изменения его диаметра;

-- температура пузырька однородная и остается равной его начальному значению, кроме температуры на бесконечно тонком поверхностном слое, который полностью испаряется с поверхности пузырька;

-- поверхность пузырька находится в паро--жидкостном равновесии: раствор фильтрата послеспиртовой барды находится под давлением насыщения;
- используется число Льюиса ($Le{ =}{Sc}/{Pr}{ =}{\alpha }/{D}{ =1}$), что позволяет объединить уравнения непрерывности для энергии и массовой доли без расчета коэффициент диффузии.
К математической постановке задачи моделирования [\ldots] привлекаются следующие уравнения:

-- уравнение непрерывности массы для пузырька воздуха (поток пара через поверхность элементарного объема):
\begin{equation} \label{l1} 
m_{{ п}}{ =4}\pi r^{{ 2}}{\rho }_{{ ф}}\nu ,  
\end{equation} 
где $\rho $ -- плотность, ${{ кг}}/{{{ м}}^{{ 3}}}$; $\nu $ -- радиальная скорость, ${{ м}}/{{ с}}$; $r$ -- радиальная координата, ${ м}$; $m$ -- массовый расход, ${{ кг}}/{{ с}}$; индексы: ${ п}$ -- пузырьки воздуха; ${ ф}$ -- фильтрат барды
или
\begin{equation} \label{l2} 
m_{{ п}}{ =4}\pi R^{{ 2}}m_{{ п}s},  
\end{equation} 
где $m_{{ пs}}$ -- массовый расход пара через поверхность пузырька ($s$, ${{ м}}^{{ 2}}$), ${{ кг}}/{{{ м}}^{{ 2}}{ с}}$; $R$ -- радиус пузырька, ${ м}$.

-- уравнения непрерывности для энергии по закону Фурье и фаз содержащихся в пузырьке по закону Фика:
\begin{equation} \label{l3} 
{\rho }_{{ ф}}c_{{ ф}}\frac{\partial T}{\partial t}{ =}{\lambda }_{{ ф}}{\nabla }^{{ 2}}T,  
\end{equation} 
\begin{equation} \label{l4} 
{\rho }_{{ ф}}\frac{\partial Y_{{ п}}}{\partial t}{ =}{\rho }_{{ ф}}D_{{ п}}{\nabla }^{{ 2}}Y_{{ п}},  
\end{equation} 
где $c$ -- теплоемкость при постоянном давлении, ${ кДж/}{ кг}\cdot { \rm K}$; $\lambda $ -- коэффициент теплопроводности, ${{ Вт}}/{{ (м}\cdot { \rm K)}}$; $T$ -- температура, ${ \rm\ K}$; $t$ -- время, с; $D$ -- коэффициент диффузии пара, ${{{ м}}^{{ 2}}}/{{ с}}$; $Y$ -- массовая доля, ${{ кг}}/{{ кг}}$.

В стационарном режиме, уравнения непрерывности имеют вид:
\begin{equation} \label{l5} 
{\lambda }_{{ ф}}\left[\frac{{ 1}}{r^{{ 2}}}\frac{\partial }{\partial r}\left(r^{{ 2}}\frac{\partial T}{\partial r}\right)\right]{ -}{\rho }_{{ ф}}c_{{ ф}}\nu \frac{\partial T}{\partial r}{ =0},  
\end{equation} 
\begin{equation} \label{l6} 
{\rho }_{{ ф}}D_{{ п}}\left[\frac{{ 1}}{r^{{ 2}}}\frac{\partial }{\partial r}\left(r^{{ 2}}\frac{\partial Y_{{ п}}}{\partial r}\right)\right]{ -}{\rho }_{{ ф}}\nu \frac{\partial Y_{{ п}}}{\partial r}{ =0},  
\end{equation} 
С учетом \eqref{l1} уравнения\eqref{l5}, \eqref{l6} записаны следующим образом:
\begin{equation} \label{l7} 
{\lambda }_{{ ф}}\frac{\partial }{\partial r}\left(r^{{ 2}}\frac{\partial T}{\partial r}\right){ -}\frac{m_{{ п}}}{{ 4}\pi }c_{{ ф}}\frac{\partial T}{\partial r}{ =0},  
\end{equation} 
\begin{equation} \label{l8} 
{\rho }_{{ ф}}D_{{ п}}\frac{\partial }{\partial r}\left(r^{{ 2}}\frac{\partial T}{\partial r}\right){ -}\frac{m_{{ п}}}{{ 4}\pi }\frac{\partial Y_{{ п}}}{\partial r}{ =0},  
\end{equation} 

Используя значение массового расхода влаги, испарившейся из фильтрата барды через единицу площади поверхности пузырька в уравнении \eqref{l2}, уравнения непрерывности приведены к безразмерному виду:
\begin{equation} \label{ZEqnNum799235} 
{\rho }_{{ ф}}{\alpha }_{{ ф}}\frac{\partial }{\partial r}\left(r^{{ 2}}\frac{\partial b_T}{\partial r}\right){ -}m_{{ п}s}R^{{ 2}}\frac{\partial b_T}{\partial r}{ =0},  
\end{equation} 
\begin{equation} \label{ZEqnNum139474} 
{\rho }_{{ ф}}D_{{ п}}\frac{\partial }{\partial r}\left(r^{{ 2}}\frac{\partial b_D}{\partial r}\right){ -}m_{{ п}s}R^{{ 2}}\frac{\partial b_D}{\partial r}{ =0},  
\end{equation} 
где ${b}_T{ =}\dfrac{c_{{ ф}}\left(T_{\infty }{ -}T\right)}{L{ +}c_{{ ф}}\left(T_s{ -}T_r\right)}$ -- безразмерная температура; $b_D{ =}\dfrac{Y_{{ п}\infty }{ -}Y_{{ п}}}{Y_{{ п}s}{ -}Y_{{ п}r}}$ -- безразмерная массовая доля; $\alpha { =}\dfrac{\lambda }{\rho c}$ -- коэффициент температуропроводности, ${{{ м}}^{{ 2}}}/{{ с}}$; $L$ -- энтальпия испарения, ${{ кДж}}/{{ кг}}$; индексы:$r$ -- внутренняя часть пузырька ограниченная радиальной координатой; $\infty $--окружающая среда пузырька (фильтрат барды).

Профили температуры и массовой доли на поверхности пузырька при переходе через элементарный объём представлены на \cref{fig:math_profile_elementary}.

\begin{figure}[htb]
\center
\includegraphics[width=0.5\textwidth]{figures/math_profile_elementary.eps}
\caption{Профили температуры и массовой доли на поверхности пузырька}\label{fig:math_profile_elementary}
\end{figure}

Для решения математической модели было предложено двойное интегрирование уравнений \eqref{ZEqnNum799235}, \eqref{ZEqnNum139474}, с граничными условиями (рисунок~\ref{fig:math_border_conditions}) по температуре:
\begin{displaymath}
\text{при} \left\{ \begin{array}{l}
r{ =}R,\quad T{ =}T_s\; и\; m_{{ п}}\cdot q{ =}{\lambda }_{{ ф}}{\left.\dfrac{\partial T}{\partial r}\right|}_s \\ 
r{ =}r_{\infty }\quad T{ =}T_{\infty } \end{array}
\right.\label{3)},
\end{displaymath} 

и для массовой доли паро--газовой смеси в пузырьке:
\begin{displaymath}
при \left\{ \begin{array}{l}
r{ =}R\quad Y_{{ п}}{ =}Y_{{ п}s}\; и\; m_{{ п}s}\cdot Y_{{ п}r}{ =}m_{{ п}s}\cdot Y_{{ п}s}{ +}\left({ -}{\rho }_{{ ф}}D_{{ п}}{\left.\dfrac{\partial Y_{{ п}}}{\partial r}\right|}_s\right) \\ 
r{ =}r_{\infty }\quad Y_{{ п}}{ =}Y_{{ п}\infty } \end{array}
\right., \label{ZEqnNum881958}
\end{displaymath}
где $q{ =}I{ +}c\left(T_{{ s}}{ -}T_r\right)$ -- количество теплоты необходимого для испарения бесконечно малого слоя фильтрата барды при $T_r$, ${{ кДж}}/{{ кмоль}}$.

\begin{figure}[htb]
\center
\includegraphics[width=0.5\textwidth]{figures/math_border_conditions.eps}
\caption{Граничные условия математической задачи моделирования процесса барботажного выпаривания}\label{fig:math_border_conditions}
\end{figure}

После интегрирования уравнение \eqref{ZEqnNum799235} было приведено к виду:
\begin{equation} \label{ZEqnNum366285} 
{\lambda }_{{ ф}}r^{{ 2}}\frac{\partial T}{\partial r}{ -}m_{{ п}s}R^{{ 2}}c_{{ ф}}T{ =}C_{{ 1}},  
\end{equation} 
где $C_{{ 1}}$ -- константа интегрирования при $r{ =}R$:
\begin{equation} \label{6)} 
C_{{ 1}}{ =}m_{{ п}s}R^{{ 2}}\left(q{ -}c_{{ ф}}T_{{ s}}\right) 
\end{equation} 
Уравнение \eqref{ZEqnNum366285} было представлено следующим образом:
\begin{equation} \label{7)} 
\frac{dT}{q{ +}c_{{ ф}}\left(T{ -}T_{{ s}}\right)}{ =}\frac{m_{{ п}s}R^{{ 2}}}{{\lambda }_{{ ф}}r^{{ 2}}},  
\end{equation} 
интегрируя которое, получено:
\begin{equation} \label{ZEqnNum555366} 
{ ln}\left(q{ +}c_{{ ф}}\left(T{ -}T_{{ s}}\right)\right){ =-}\frac{m_{{ п}s}R^{{ 2}}}{{\rho }_{{ ф}}{\alpha }_{{ ф}}}\frac{{ 1}}{r}{ +}C_{{ 2}},  
\end{equation} 
где $C_{{ 2}}$ -- константа интегрирования при $r{ =}\infty $:
\begin{equation} \label{9)} 
C_{{ 2}}{ =ln}\left(q{ +}c_{{ ф}}\left(T_{\infty }{ -}T_{{ s}}\right)\right) 
\end{equation} 

Используя \eqref{ZEqnNum555366} при $r{ =}R$, уравнение \eqref{ZEqnNum799235} для выражения массового расхода пара на поверхности пузырька принимает вид:
\begin{equation} \label{10)} 
m_{{ п}s}{ =}\frac{{\rho }_{{ ф}}{\alpha }_{{ ф}}}{R}{ ln}\left(\frac{c_{{ ф}}\left(T_{\infty }{ -}T_{{ s}}\right)}{I{ +}c_{{ п}}\left(T_{{ s}}{ -}T_r\right)}{ +1}\right) 
\end{equation} 
или
\begin{equation} \label{ZEqnNum991372} 
m_{{ п}s}{ =}\frac{{\rho }_{{ ф}}{\alpha }_{{ ф}}}{R}{ \ln}\left(b_T{ +1}\right) 
\end{equation} 

После аналогичных вычислений уравнение \eqref{ZEqnNum139474} с граничными условиями \eqref{ZEqnNum881958} было преобразовано к виду:
\begin{equation} \label{12)} 
m_{{ п}s}{ =}\frac{{\rho }_{{ ф}}D_{{ п}}}{R}{ \ln}\left(\frac{Y_{{{ п}}_{\infty }}{ -}Y_{{ п}}}{Y_{{ п}s}{ -}Y_{{ п}r}}{ +1}\right) 
\end{equation} 
или
\begin{equation} \label{ZEqnNum341635} 
m_{{ п}s}{ =}\frac{{\rho }_{{ ф}}D_{{ п}}}{R}{ \ln}\left(b_D{ +1}\right),  
\end{equation} 
где $b_T$ и $b_D$ числа тепло-- и массопереноса соответственно.

Решение системы уравнений \eqref{ZEqnNum799235}--\eqref{ZEqnNum341635} с двумя неизвестными основано на зависимости $b_T{ =}b_D$. После сопоставления уравнений \eqref{ZEqnNum991372}, \eqref{ZEqnNum341635} определяли $T_s$ и $Y_{{ п}s}$, используя относительное значение температуры $T_{{ отн}}$ методом приближений:
\begin{equation} \label{14)} 
T_{{ отн}}{ =}T_s{ +}\frac{T_{\infty }{ -}T_s}{{ 3}} 
\end{equation} 
и
\begin{equation} \label{ZEqnNum400623} 
Y_{{ п.отн}}{ =}Y_{{ п}s}{ +}\frac{Y_{{ ф}\infty }{ -}Y_{{ п}s}}{{ 3}} 
\end{equation} 
Из уравнения \eqref{ZEqnNum400623}:
\begin{equation} \label{16)} 
Y_{{ п}s}{ =}\frac{{ 1}}{\left(\frac{P_{\infty }}{P_{{ н}}\left(T_s\right)}{ -}{ 1}\right)\frac{M_{{ в}}}{M_{{ п}}}{ +1}},  
\end{equation} 
где $P$ -- парциальное давление, Па; $M$ -- молекулярная масса, $\dfrac{г}{моль}$;  ${\chi }_{{ пs}}{ =}\dfrac{Y_{{ пs}}M_{{ ф}}}{M_{{ п}}}$ -- мольная доля пара на поверхности пузырька; $M_{{ ф}}{ =}\left({ 1-}{\chi }_{{ пs}}\right)M_{{ в}}{ +}{\chi }_{{ пs}}M_{{ п}}$ -- молекулярная масса фильтрата, $\dfrac{г}{моль}$; индексы: в -- воздух; н -- насыщение.

Для нахождения $T_s$ определяли коэффициент теплоемкость и теплопроводности паро--газовой фазы пузырька по функциональным зависимостям, учитывающих сумму долей пара, испарившегося из фильтрата барды и воздуха в пузырьке:
\begin{equation} \label{ZEqnNum894908} 
c_{{ п}}\left(T_s\right){ =}\left({ 1-}Y_{{ отн}}\left(T_s\right)\right)c_{{ ф}}{ \ \ }T_{{ отн}}\left(T_s\right){ +}Y_{{ отн}}\left(T_s\right)c_{{ в}}\left(T_{{ отн}}\left(T_s\right)\right) 
\end{equation} 
\begin{equation} \label{ZEqnNum280770} 
{\alpha }_{{ п}}\left(T_s\right){ =}\left({ 1-}Y_{{ отн}}\left(T_s\right)\right){\alpha }_{{ ф}}{ \ \ }T_{{ отн}}\left(T_s\right){ +}Y_{{ отн}}\left(T_s\right){\alpha }_{{ в}}\left(T_{{ отн}}\left(T_s\right)\right) 
\end{equation} 

Законы изменения (17), (18) подбирали методом машинного эксперимента, обеспечивающего максимальное сближение расчетных и экспериментальных данных.

Для идентификации модели проводили экспериментальное исследование процесса барботажного выпаривания фильтрата барды на опытной барботажной колонне диаметром ${ 15,3}$ см, высотой ${ 1,1}$ м, снабженной ${ 2}$ кВт нагревателем воздуха. 

Барботажное устройство -- перфорированная пластина алюминия с ${ 15}$ отверстиями диаметром ${ 2}\cdot { 1}0^{{ -}{ 3}}$ м. 
Скорость потока воздуха на входе, поддерживали постоянным в течение всего времени и измеряли с помощью калиброванного ротаметра.

\begin{figure}[b!]
\centering
\begin{subfigure}{0.45\linewidth}
\centering
\includegraphics{math_graph1}
\caption{}\label{fig:math_results1}
\label{fig:main_view_membrane1}
\end{subfigure}
\quad
\begin{subfigure}{0.45\linewidth}
\centering
\includegraphics{math_graph2}
\caption{}\label{fig:math_results2}
\label{fig:main_view_membrane2}
\end{subfigure}
\quad
\begin{subfigure}{0.45\linewidth}
\centering
\includegraphics{math_graph3}
\caption{}\label{fig:math_results3}
\label{fig:main_view_membrane3}
\end{subfigure}
\quad
\begin{subfigure}{0.45\linewidth}
\centering
\includegraphics{math_graph3}
\caption{}\label{fig:math_results4}
\label{fig:main_view_membrane4}
\end{subfigure}
\caption[Результаты моделирования]{Результаты моделирования: зависимости продолжительности процесса $t$ от скорости испарения $m_{{ пs}}$ (а), массовой концентрации $w$ (б), температуры фильтрата $T$, высоты барботирования $H$ (в) и радиуса пузырька (г)}
\end{figure}


Эксперимент проводили в соответствии со следующей методикой: взвешенное количество фильтрата вносили в колонну и, на протяжении всего времени барботирования периодически считывали значения температуры и общей высоты жидкости, температуру воздуха на входе, а также массы конденсата. 
Опыт проводили до момента наступления квазистационарного состояния процесса, при котором температура жидкости и скорость испарения оставались постоянными.

Условия моделирования: массовая концентрация фильтрата ${ 1200-1400}$ \(мг/л\), объем фильтрата ${ 10-12}$ л, температура фильтрата ${ 330-335}$
K, температура воздуха ${ 850-87 }$ K, скорость воздуха ${ 0,01-0,0}{ 2}$ \(м/с\), высота барботирования ${ 0,20-0,25}$ м. Получены зависимости продолжительности процесса $t$ от скорости испарения $m_{{ пs}}$ (рисунок~\ref{fig:math_results1}), массовой концентрации $w$ (рисунок~\ref{fig:math_results2}), температуры фильтрата $T$, высоты барботирования $H$ (рисунок~\ref{fig:math_results3}) и радиуса пузырька (рисунок~\ref{fig:math_results4}).

Отклонение расчетных и экспериментальных данных не превышало ${ 12}$ \%. 
Модель может быть использована при проектировании баботажных аппаратов и управлении технологическими параметрами в области допустимых технологических свойств целевого продукта.

Таким образом, предлагаемый метод моделирования открывает возможности численного определения массового и теплового потока на поверхности пузырька, а также решать задачи рационального использования энергии в зависимости от производительности барботажного аппарата, и его геометрических размеров.
