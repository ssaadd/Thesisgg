\subsection{Газосодержание в неизотермической системе}

Значение газосодержания может быть получено из уравнения:
\begin{equation}
\frac{V_g}{V_{L}}=\frac{\varepsilon}{1-\varepsilon}\label{eq:math_holdup}
\end{equation}

Объем жидкой фазы (\(V_{L}\)) для гипотетического процесса где нет тепло- и массопереноса принимается постоянным. 
Объем газовой фазы (\(V_{g}\)) находят по уравнению:
\begin{equation}
  {V}_{g}=\stackrel{-}{V}{f}_{orif}N{t}_{r}=\stackrel{-}{V}f{t}_{r}\label{eq:math_gasvolume} 
\end{equation}
где \(V\)~"--- средний объем пузырьков, \(м^{3}\);
\(f\)~"--- частота образования пузырьков, \(1/с\);
\(N\)~"--- число отверстий;
\({t}_{r}\)~"--- среднее время пребывание пузырьков в колонне, \(с\);
нижний индекс \({orif}\)~"--- отверстие.

Далее из \cref{{eq:math_holdup},{eq:math_gasvolume}} для гипотетического процесса без тепло- массопереноса, газосодержание находят по уравнению: \begin{equation}
 \frac{\varepsilon }{1-\varepsilon }= \frac{ \stackrel{-}{V}{f}_{orif}{t}_{r}}{\left({\stackrel{-}{V}f}_{orif}{t}_{r}\right)_{hyp}}\left(\frac{\varepsilon}{1-{\varepsilon }}\right)_{hyp} 
\end{equation}
где \(\varepsilon\)~"--- газосодержание;
нижний индекс \({hyp}\)~"--- гипотетический.

Объем и время пребывания пузырька в колонне может быть
расчитываться путем введения поправочный коэффициента по уравнению:
\begin{equation} 
\frac{\varepsilon }{1-\varepsilon }=\overline{\beta}^{3} \frac{{t}_{r}}{{t}_{{r}_{hyp }}}\frac{{\varepsilon }}{1-{\varepsilon }_{hyp}} \end{equation}
где \(\overline{\beta}\)~"--- средний безразмерный радиус пузырька во время стадии подъема в колонне, \(R(t)/R_{F}\), находят по уравнению:
\begin{equation} 
 \overline{\beta}=\frac{1}{Z}\underset{0}{\overset{Z}{\int }}\beta \left(z\right)dz
\end{equation} 

При оценке времени нахождения пузырька предполагается однородный режим барботирования, тогда для гипотетического случая время определяют из зависимости: \(Z/U_{hyp}+t_{F}\), где \(U_{hyp}\)~"--- скорость подъема пузырька по отношению к неподвижной системе отсчета,
определяется с использованием радиуса образования пузырька, \(R_{f}\); \(t_{f}\)~"--- время  образования пузырьков (\(t_{F}=1/f_{orif}\)).
Таким образом, для барботажной колонны полу-периодического действия, время нахождения пузырька рассчитывается по уравнению: 
\begin{equation}  
 z\left({t}_{r}\right)=\underset{{t}_{F}}{\overset{{t}_{r}}{\int }}Udt=Z 
\end{equation}
где \(U\)~"--- скорость подъёма пузырька, \(м/с\), аппроксимируется конечной скоростью пузырька.

