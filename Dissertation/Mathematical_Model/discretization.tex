\subsection{Временная дискретизация}

Группируя вместе все члены, кроме нестационарного,
в уравнении~\eqref{eq:math_int_divergent}  в поток векторного поля \(F(\phi)\), схема первого порядка явной временной дискретизации может быть выражена следующим образом:
\begin{equation}
 \frac{{\phi }^{n+1}-{\phi }^{n}}{\Delta f}=F\left({\phi }^{n}\right) 
\end{equation}
где \(\Delta t\)~"--- временной шаг дискретизации. 

Уравнение является параболическим по временной переменной. %, поэтому обновление временного шага может быть получено по схемеtime-marching.
При этом временной шаг установлен с определенным пределом, чтобы избежать численной неустойчивости во время расчета.

\subsection{Пространственная дискретизация}

Выбор техники пространственной дискретизации определяет схему интерполяции переменных в центре ячеек на центрах граней.
Более точной из доступных является схема \textit{Monotonic Upstream-centered Scheme for Conservation Laws}
(MUSCL). 
Это объединение схемы центральной конечной разности  (CD) со схемой  разности против потока второго порядка (SOU):
\begin{equation}
 {\phi }_{f}=\theta {\phi }_{f,CD}+\left(1-\theta \right){\phi }_{f,SOU} 
\end{equation}
где \(\theta\)~"--- коэффициент объединения.

Значение на центрах граней полученных по CD схеме:
\begin{equation}
{\phi }_{f,CD}=  \frac{1}{2}\left({\phi }_{0}+{\phi }_{1}\right)+\frac{1}{2}\left[{\left(\nabla \phi \right)}_{0}{r}_{0}+{\left(\nabla \phi \right)}_{1}{r}_{1}\right] 
\end{equation}
где индексы \(0\) и \(1\) относятся к ячейкам, которые разделяются гранью \(f\); \(\bigtriangledown \phi\)~"--- дискретизация центром градиента клеток; \(\boldsymbol r\)~"--- вектор, направленный из центра тяжести на центр тяжести грани, см.~\cref{fig:math_control_volume}.

Значение на центрах граней полученных по SOU схеме:
\begin{equation}
{\phi }_{f,SOU}= \phi +\nabla \phi r 
\end{equation}
где все субъекты относятся к потоку ячейки.

Градиент центра ячейки на грани в уравнении~\eqref{eq:math_int_divergent} всегда дискретизируется по схеме CD:
\begin{equation}
 {\left(\nabla \phi \right)}_{f} = \frac{1}{2}\left[{\left(\nabla \phi \right)}_{0}+{\left(\nabla \phi \right)}_{1}\right]
\end{equation}
