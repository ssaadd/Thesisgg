\subsection{Метод решения дифференциальных уравнений в ANSYS Fluent}

Fluent использует метод конечных объемов для преобразования нелинейных частных дифференциальных уравнений непрерывности в линейные алгебраические уравнения, которые могут быть решены численно. 
Уравнение переноса для общей переменной \(\phi\):
\begin{equation}
 \frac{\partial \left(\rho \phi \right)}{\partial t}+\nabla \left(\rho \phi u\right)=\nabla \left(\Gamma \nabla \phi \right)+{S}_{\phi } 
\end{equation}
интегрируется по контрольному объему \(V\) и граничной площади \(A\):
\begin{equation}
 \underset{V}{\int }\frac{\partial \left(\rho \phi \right)}{\partial t}dV+\underset{A}{\int }\rho \phi u\cdot ndA=\underset{A}{\int }\Gamma \nabla \phi \cdot ndA+\underset{V}{\int }S _{\phi}dV \label{eq:math_int_divergent}
\end{equation}

В этом уравнении, для преобразования объемного интеграла дивергенции векторного поля  в поверхностный интеграл от скалярного произведения между векторным полем и исходящей единичной нормалью \(\boldsymbol n\) к граничной ячейке, была применена теорема дивергенции.
 
При условии, что значение интегрируемой функции в центре ячейки равно среднему значению функции в ячейке, объемный интеграл в уравнении~\eqref{eq:math_int_divergent} может быть аппроксимирован умножением значения подынтегральной функции в центре ячейки по её объему.
Кроме того, полагая, что значение подынтегральной функции в центре клетки равна средней функции клетки, интегралы поверхности в уравнении~\eqref{eq:math_int_divergent} могут быть аппроксимированы правилом середины,
путем умножения значения подынтегральной функции в каждом центре ячейки на
площадь \(A_{f}\): 
\begin{equation}
\underbrace{\frac{\partial \left(\rho \phi \right)}{\partial t}V}_{\text{нестационарный~элемент}}+ \underbrace{\sum _{f}^{{N}_{f}} \rho_{f} \phi_{f} \boldsymbol u_{f}\cdot \boldsymbol n_{f}A}_{\text{конвективный~элемент}}= \underbrace{\sum _{f}^{{N}_{f}}\Gamma_{f} \left(\nabla \phi\right)_{f} \cdot \boldsymbol n_{f}A_{f}}_{\text{диффузионный~элемент}}+\underbrace{S _{\phi}V}_{\text{источник}}
\end{equation}
где нижний индекс \(f\) у членов уравнения относит их к гранеобразующим, в то время как величины без индекса \(f\)~"--- переменные
центра ячеек. 
Сумма граней \(N_{f}\) контрольного объема изображенного на рисунке~\ref{fig:math_control_volume}: ячейка \(0\) имеет \(N_{f}=3\), а ячейка \(1\)~"---
 \(N_{f} = 4\). 
\begin{figure}
\centering
\input{figures/control_volume.pdf_tex}
\caption{Контрольный объем для дискретизации основного уравнения непрерывности}\label{fig:math_control_volume}
\end{figure}
Fluent использует совмещенным технику, при которой алгебраические уравнения решаются для переменных в центре ячеек, таким образом, выбирается такая схема интерполяции, чтобы выразить гранецентрированные переменные в уравнении~\eqref{eq:math_int_divergent}, как функции значений в центре ячеек. 
Кроме того, необходимо обеспечить схему пространственной дискретизации
градиентов. 
