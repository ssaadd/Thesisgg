\subsection{Итоговые алгебраические уравнения}

Общее~\cref{eq:math_int_divergent} содержит неизвестное значение \(\phi\) в центре ячейки где \cref{eq:math_int_divergent}
дискретизировано, а также неизвестные значения в центрах соседних ячеек.
Полученное уравнение можно, как правило, является нелинейным, поэтому линеаризации уравнения является частью алгоритма решения и окончательное линейное алгебраическое уравнение для каждой ячейки домена имеет следующий вид:
\begin{equation}
a{\phi }^{n+1}=\sum_{nb}{a}_{nb}{\phi }_{nb}^{n+1}+b
\end{equation}
где индекс \(nb\) относится к соседним клеткам; \(a\) и \(a_{nb}\)~--- коэффициенns линеаризации для \(\phi\) и \(\phi_{nb}\) ; \(b\)~--- известный член уравнения. 
Для каждого уравнения потока, линейная система уравнений \(N\) должна быть решена, с \(N\) числом расчетных ячеек.
Численное решение каждого уравнение потока, таким образом, состоит в решении линейной системы:
\begin{equation}
\boldsymbol A{\phi }^{n+1}=b
\end{equation}
где \(\boldsymbol A\)~"--- \([N\times N]\) коэффициент разреженной матрицы, \(\phi\)~"--- \([N\times 1]\) неизвестные члены вектора; \(\boldsymbol b\)~"--- \([N\times 1]\) известные, члены вектора. 
