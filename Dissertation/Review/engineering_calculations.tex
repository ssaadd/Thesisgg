\section{Инженерные методы расчета}

\subsection{Вакуум-выпарной аппарат}\label{seq:calculation_vacuum}

Количество выпариваемой воды, \(\text{кг}/\text{ч}\) определяют по уравнению:
\begin{equation}
 W={G}_{\text{н}}\left(1-\frac{{w}_{\text{н}}}{{w}_{\text{к}}}\right), 
\end{equation}
где \({G}\)~"--- производительность по жидкости, \(\text{кг}/\text{ч}\); \(w\)~"--- влажность; индексы: \(\text{н}\)~"--- начальное значение; \(\text{к}\)~"---
конечное значение.

Расход теплоты на выпаривание \(Q\) включает затраты тепла на нагрев \(Q_{\text{нагр}}\), испарение \(Q_{\text{исп}}\), дегидратацию и потери в окружающую
среду (принимаются равными \(10\)\% от \(\left(Q_{\text{нагр}} + Q_{\text{исп}}\right)\)).

Расход тепла на нагрев, \(\text{ккал}/\text{ч}\) находят по уравнению:
\begin{equation}
 {Q}_{\text{нагр}}={G}_{\text{н}}{c}_{\text{н}}\left({t}_{\text{кип}}-{t}_{\text{н}}\right),
\end{equation}
где \(t\)~"--- температура, \(\celcius\); индексы: \(\text{кип}\)~"--- кипение.

Расход тепла на испарение, \(\text{ккал}/\text{ч}\):
\begin{equation}
 {Q}_{\text{исп}}=Wr,
\end{equation}
где \(r\)~"--- теплота парообразования, \(\text{ккал}/\text{кг}\).

Общий расход тепла в аппарате, \(\text{ккал}/\text{ч}\):
\begin{equation}
 Q=\left({Q}_{\text{нагр}}+{Q}_{\text{исп}}\right)\cdot 1,1 
\end{equation}

Расход греющего пара, \(\text{кг}/\text{ч}\):
\begin{equation}
D=\frac{Q}{{r}_{\text{г. п.}}},
\end{equation}
где \({r}_{\text{г. п.}}\)~"--- теплота конденсации греющего пара, \(\text{ккал}/\text{кг}\).

Удельный расход пара, \(\text{кг}/\text{кг}\):
\begin{equation}
 d=\frac{D}{W} 
\end{equation}

Поверхность нагрева, \(\text{м}^{2}\):
\begin{equation}
 F=\frac{Q}{{K}_{\Delta }{t}_{\text{пол}}},
\end{equation}
где \(K\)~"--- коэффициент теплопередачи в теплообменнике, \(\text{ккал}/\left(\text{м}^{2}\text{ч}\celcius\right)\); 
\(\Delta {t}_{\text{пол}}={t}_{\text{г.п.}}-{t}_{\text{кип}} \)~"--- полезная разность температур, \(\celcius\)


\subsection{Барботажный выпарной аппарат}

Теоретическое количество воздуха \(\text{кг}/\text{кг}\), необходимое для сжигания \(1\) \(\text{м}^{3}\) природного газа:
\begin{equation}
 {L}_{0}=9,4\frac{{\rho }_{\text{в}}}{{\rho }_{\text{т}}}, 
\end{equation}
где \(\rho\)~"--- плотность, \(\text{кг}/\text{м}^{3}\); индексы: \(\text{в}\)~"--- воздух; \(\text{т}\)~"--- топливо.

Количество водяных паров, образующихся при сжигании 1 кг газа, \(\text{кг}/\text{кг}\):
\begin{equation}
 {G}_{\text{п}}=\sum \frac{0,09}{12m+n}{C}_{m}{H}_{n} 
\end{equation}

Количество сухих газов, образующихся при сжигании \(1\) кг природного газа, \(\text{кг}/\text{кг}\):
\begin{equation}
 {G}_{\text{с. г.}}=1+\alpha {L}_{0}-{G}_{\text{п}},
\end{equation}
где \(\alpha\)~"--- коэффициент избытка воздуха.

Количество водяных паров в топочных газах, \(\text{кг}/\text{кг}\):
\begin{equation}
 {G}_{\text{п}}^{\prime }=\frac{\alpha {d}_{0}{L}_{0}}{1000}+{G}_{\text{п}},
\end{equation}
где \({d}_{0}\)~"--- влагосодержание сухого воздуха, \(\text{г}/\text{кг}\).

Влагосодержание топочных газов, \(\text{г}/\text{кг}\):
\begin{equation}
 {d}_{вх}=\frac{{G}_{1}\cdot 1000}{{G}_{\text{с. г}}} 
\end{equation}

Расход топочных газов, \(\text{кг}/\text{ч}\):
\begin{equation}
 L=\frac{Q}{{c}_{\text{вх}}{t}_{\text{вх}}-{c}_{\text{вых}}{t}_{\text{вых}}} 
\end{equation}

Расход тепла с отходящими из выпарного аппарата газами, \(\text{ккал}/\text{ч}\):
\begin{equation}
 {Q}_{1}=L{c}_{\text{от. г}}\left({t}_{\text{вых}}-{t}_{0}\right) 
\end{equation}

Суммарный расход тепла, вносимого в выпарной аппарат, \(\text{ккал}/\text{ч}\):
\begin{equation}
 \sum Q=Q+{Q}_{\text{от. г}},
\end{equation}
где \(Q\) определяют из соответствующих уравнений рассмотренных в~\cref{seq:calculation_vacuum}.

Расход топлива, \(\text{м}^{3}/\text{ч}\):
\begin{equation}
 B=\frac{\sum Q}{{Q}_{\text{н}}{\eta }_{т}},
\end{equation}
где \({Q}_{\text{н}}\)~"--- низшая теплота сгорания топлива, \(\text{ккал}/\text{ч}\); \({\eta }_{т}\)~"--- КПД топки.

Удельный расход тепла, \(\text{м}^{3}/\text{ч}\):
\begin{equation}
 q=\frac{B{Q}_{\text{н}}}{W} 
\end{equation}

Влагосодержание газов на выходе из аппарата, \(\text{г}/\text{кг газа}\):
\begin{equation}
 {d}_{\text{вых}}={d}_{\text{вх}}+\frac{W\cdot 1000}{L} 
\end{equation}

Удельный объем газов на входе/выходе, \(\text{м}^{3}/\text{ч}\):
\begin{equation}
{\nu }_{\text{вх(вых)}}^{0}=4,64\cdot {10}^{-6}\left(622+{d}_{вх(вых)}\right)\left(273+{t}_{вх(вых)}\right)
\end{equation}

Объем газов на входе/выходе,  \(\text{м}^{3}/\text{ч}\):
\begin{equation}
 {\nu }_{\text{вх(вых)}}={\nu }_{\text{вх(вых)}}^{0}L 
\end{equation}

Определение <<зеркала>> испарения аппарата, \(\text{м}^{2}\):
\begin{equation}
 F=\frac{{\nu}_{\text{вых}}}{{w}_{2}\cdot 3600},
\end{equation}
где \(w_{2}\)~"--- скорость газов в горизонтальном сечении аппарата.

Определения диаметра барботажных труб:
\begin{equation}
 F=\frac{{\nu}_{\text{вх}}}{{w}_{1}\cdot n \cdot 3600},
\end{equation}
где \(w_{1}\)~"--- скорость в барботажной трубе; \(n\)~"--- число труб.

Тогда диаметр труб, \(\text{м}\):
\begin{equation}
 d=\sqrt{\frac{4F}{\pi }} 
\end{equation}
