\clearpage
\section{Современная техника и технология производства сухой послеспиртовой барды}

В процессе производства спирта из зернового сырья образуется значительное количество отходов производства "--- послеспиртовой жидкой барды, которая при сбросе в стоки вызывает загрязнение окружающей среды. 
В то же время, барда обладает известной питательной и кормовой ценностью, поскольку именно в барде остается весь белок зерна после того, как крахмалистые компоненты переработаны на этанол. 
В сельском хозяйстве многих стран широко применяются продукты на основе барды, содержащие протеин, легкоперевариваемые углеводы, витамины, микро-- и макроэлементы. 
С ростом объемов производства этилового спирта, в том числе из-за расширения его применения в качестве биотоплива, проблема переработки послеспиртовой барды приобретает большую экологическую значимость \cite{Androsonov_2010_4} в России это подтверждается законом №~102 ФЗ, который предписывает обязательное использование линий по переработке барды производителями спирта с 1 января 2009 г. (перенесено на 1 января 2010 г. в связи с финансовым кризисом). 
Хотя проведенные исследования \cite{Shunyaeva_2004_15} показали, что слив барды до определенного предела не наносит невосполнимого ущерба почве полей фильтрации, так как в течение двух месяцев после слива наблюдается восстановление количественного и качественного составов микрофлоры грунта, при крупномасштабном производстве спирта под слив барды уходят большие территории, кроме того уничтожается довольно ценный в качестве корма для животных продукт. 
Необходимость разработки процесса переработки барды, как неоднократно отмечалось \cite{Dvoreckiy_1998,Zuzina_1990}, вызвана, прежде всего, соображениями охраны окружающей среды путем создания малоотходного энерго- и ресурсосберегающего производств. 
Таким образом, проведение работ, направленных на усовершенствование методов переработки барды, становится особенно актуальным. 
Основной трудностью в утилизации послеспиртовой барды является переработка растворимых веществ. 
Фактически, на спиртовом заводе мощностью 3000 дал образуется до 350 м$^3/$(сутки) барды, в растворимой части которой может содержаться вещества с химической потребностью в кислороде (ХПК) более 50 000 мг~O$_{2}$/л. 
В настоящее время существует несколько широко распространенных направлений по переработке послеспиртовой барды. 
Они базируются на принципах, показанных на рис. \ref{ShemaUtil}\cite{Novikov_2007_2_Journal}. 

\begin{figure} 
\centering 
\begin{small} 
\def\svgwidth{1\linewidth} 
\input{figures/shemautilizacii.pdf_tex} 
\end{small} 
\caption{Схема утилизации послеспиртовой барды} 
\label{ShemaUtil} 
\end{figure}
 
При выборе схемы для внедрения в производство необходимо знать преимущества и недостатки каждой из них. 
В настоящее время в большинстве случаев используется комбинирование тех или иных схем. Особенностью схемы с получением кормовых дрожжей является обеспечение утилизации большинства растворенных органических соединений барды и перевод их в усваиваемый кормовой белок в виде кормовых дрожжей. 
В России построен ряд заводов по выпуску сухих кормовых дрожжей, работающих на послеспиртовой барде (<<Береговской>>, <<Мариинский>>, <<Мамадышский>> и др.). 
Кормовые дрожжи "--- это то концентрированная белковая добавка к кормам, используемая на многих сельхозпредприятиях и комбикормовых заводах. Содержание белка в кормовых дрожжах может превышать 45--46\%. 
Комбинация микробного дрожжевого белка с растительным делает дрожжевой кормовой концентрат (ДКК) не просто кормовой добавкой с высоким содержанием белка, а настоящей основой кормов для свиноводства и птицеводства без диетологических ограничений, связанных с аминокислотным составом и усвоением протеинов из зернового источника \cite{web_spbarda}. 
Технологическая схема переработки барды с получением кормовых дрожжей представлена на рис. \ref{ShemaUtil2} \cite{web_spbarda}. 

\begin{figure} 
\centering 
\begin{small} 
\def\svgwidth{0.8\linewidth} 
\input{figures/ShemaKormDr.pdf_tex} 
\end{small} 
\caption{Схема с получением кормовых дрожжей} 
\label{ShemaUtil2} 
\end{figure} 

\textbf{Подготовка барды.} 

Горячая барда поступает на теплообменник и далее в аппараты ферментативного гидролиза, где происходит биохимическое обогащение барды за счёт перевода в растворимое состояние части взвешенных веществ, для дальнейшей их утилизации дрожжами. При этом, в результате ферментативного гидролиза клетчатки образуются усваиваемые дрожжами органические соединения. 
Ассимиляция этих питательных веществ делает возможным переработать небелковую часть взвешенных веществ барды в кормовые дрожжи, тем самым повысив общее содержание белка в готовой продукции и, соответственно, её питательную ценность.
 
\textbf{Прием и центрифугирование подготовленной барды.} Подготовленная барда поступает в напорную емкость и далее насосом на батарею из двух декантерных центрифуг. 
На центрифугах из суспензии барды выделяются две фракции: фракция влажных взвешенных веществ (кек) и жидкая фракция (фугат). 
Влажный кек самовыгрузкой подаётся непосредственно либо с помощью винтового конвейера в шнековый смеситель. Фугат поступает самотеком в сборник фугата (ферментный реактор). 
\textbf{Подготовка субстрата для ферментации.} 
Субстрат для ферментации (процесса выращивания кормовых дрожжей) представляет собой фугат, обогащённый питательными солями. 
Раствор питательных солей готовится в общем сборнике,который поступают растворы из отдельных расходных емкостей. 
Питательный раствор солей из общего сборника самотеком подается в сборник фугата. 
Затем полученная смесь насосом подается на первую ступень ферментации. 
\textbf{Ферментация}: 

\begin{itemize} 
\item подготовка чистой культуры дрожжей. На первой стадии чистая культура из пробирки выращивается стандартным способом с использованием дрожжанок. Полученную дрожжевую суспензию подают в ферментатор, где её доводят до требуемого объёма; 
\item первая ступень ферментации. 
Смесь фугата и питательных солей из смесительного сборника подаётся в ферментатор, через который производится барботаж воздуха от воздуходувки. 
Температура ферментации поддерживается постоянной при помощи рециркуляции дрожжевой суспензии насосом через внешний теплообменник. 
Заданный уровень кислотности поддерживается путём добавки серной кислоты. 
Вспененная дрожжевая суспензия из ферментатора самотеком поступает во флотатор, где происходит отделение сгущенной биомассы дрожжей от дрожжевой бражки. 
Дрожжевая бражка из флотатора насосом подается на ферментатор 2 ступени; 
\item вторая ступень ферментации. Отфлотированная дрожжевая бражка из флотатора подается на вторую ступень ферментации. 
Во втором ферментаторе бражка барботируется воздухом. 
Температура ферментации поддерживается постоянной при помощи рециркуляции дрожжевой суспензии насосом через внешний теплообменник. 
Заданный уровень кислотности поддерживается путём добавки серной кислоты. 
Вспененная дрожжевая суспензия из ферментатора самотеком поступает во флотатор, где происходит отделение сгущенной биомассы дрожжей от дрожжевой бражки. 
Жидкая фаза дрожжевой суспензии отбирается и поступает на доочистку на установки мембранной фильтрации или непосредственно на очистные сооружения. 
Часть жидкой фазы может быть возвращена в технологию производства спирта на участок замеса. 
\end{itemize} 

\textbf{Флотация.} 
Во флотатор поступает вспененная дрожжевая суспензия с обеих ступеней ферментации. 
Путём флотации дрожжевая суспензия разделяется на флотоконцентрат с содержанием дрожжевой биомассы и жидкую фазу -- дрожжевую бражку. 
Флотоконцентрат насосом подаётся на сепарацию, а бражка "--- на мембранные установки или на очистные сооружения. \textbf{Сепарация дрожжей.} 
На сепараторах происходит дальнейшее сгущение флотоконцентрата. 
Сгущенный флотоконцентрат поступает на барабанный вакуум-фильтр, а сепарированная дрожжевая бражка "--- на мембранные установки или на очистные сооружения. 
На барабанном вакуум-фильтре производится окончательное сгущение дрожжей, после чего дрожжевая масса срезается с полотна вакуум-фильтра и подаётся в шнековый смеситель. 
Отделённая бражка водокольцевым вакуумным насосом подаётся на мембранные установки или на очистные сооружения. 
\textbf{Получение ДКК.} 
Отделенный на декантерной центрифуге кек и сгущенные на вакуум-фильтре дрожжи поступают в шнековый смеситель, в котором в течение 8-ми мин происходит их непрерывное перемешивание до однородной комкующейся массы. 
Влажная смесь кека и дрожжей подается транспортером в сушильную роторно-трубчатую печь. 
В сушильной печи происходит окончательная сушка продукции. \textbf{Очистка стоков.} 
В процессе производства обеспечивается частично замкнутый цикл водопользования, когда очищенная вода может быть возвращена в производственный процесс или может быть сброшена в имеющиеся очистные сооружения. 
Однако и в случае использования процесса производства кормовых дрожжей концентрация органических веществ в сточных водах весьма существенна. 
Радикальным способом, позволяющим решить указанные проблемы, является совмещение химико-технологического процесса производства этанола с биохимическим процессом очистки стоков при использовании бактерий, например, рода \textit{Pseuas}, которые обеспечивают существенно большую степень конверсии органических веществ данных стоков. 
Побочным продуктом (отходом основного производства) является послеспиртовая барда, которая может быть использована в качестве субстрата. 
Проектируемое производство должно включать два основных технологических процесса "--- переработку биомассы и очистку стоков с одновременным отделением продукта от жидкой фазы. 
Весь комплексный технологический процесс в результате получается практически безотходным за счёт конверсии существенной части органических веществ, содержащихся в стоках. 
Кроме этого, в результате получается продукт, который может быть использован в качестве кормового компонента. 
В основу предлагаемой технологии положен трехстадийный аэробный процесс непрерывного выращивания специально подобранных микроорганизмо в ферментерах интенсивного массообмена, при использовании органических веществ, содержащихся в послеспиртовой барде, лютерных и промывных водах в качестве единственного источника углерода \cite{Novikov_2007_2_Journal}. 
Несмотря на достаточно высокое энергопотребление, применение также находят схемы с выпарными станциями и сушкой (DDGS "--- Dried Distillers Grains with Solubles), представленные на рис.\ref{ShemaDDGS} \cite{web_distil}. 
Технология упаривания жидкой части барды после удаления взвешенных веществ на выпарных станциях -- самая распространенная в мире. 
Аппаратурные решения (один из вариантов "--- на схеме ниже) на российском рынке предлагаются шведской компанией <<Alfa-Laval>>, датской <<Atlas-Stord>>, рядом китайских и российских компаний (так называемая <<Китайская схема>>) и др. 

\begin{figure} 
\centering 
\begin{small} 
\def\svgwidth{0.8\linewidth}
\input{figures/ShemaDDGS.pdf_tex} 
\end{small} 
\caption{Схема переработки послеспиртовой барды с использованием выпарных станций} 
\label{ShemaDDGS} 
\end{figure}
 
Привлекательная простота технического оформления не снимает, однако, целого ряда проблем: стоимость выпарных станций и вспомогательного оборудования достаточно высока; процесс выпарки требует значительных энергетических затрат (порядка 1500 кВт$\cdot$~ч на кубометр фильтрата/фугата), а утилизация получаемого конденсата с ХПК 1500--3000~мг~О$_{2}$/л становится отдельной задачей, решение которой внутри технологии DDGS не заложено. 
В России полный цикл переработки барды в DDGS полностью реализован только на одном предприятии (СЗ <<Буинский>>, Татспиртпром). 
На ряде спиртовых заводов (<<Уржумский>>, <<Корыстово>> и др.) реализован усечённый цикл переработки барды в продукт DDGS. 
В этом случае перерабатывается только твёрдая фаза барды, а жидкий раствор сливается (данных о его утилизации не имеется) \cite{web_spbarda}. 
Схемы с производством биогаза не нашли широкого применения в России. 
В основном, данная технология применяется для переработки мелассной барды. 
Технология переработки барды на биогаз основана на брожении без доступа кислорода. 
Барда подаётся в специальные емкости, где в активном состоянии поддерживается масса анаэробных бактерий.
Бактерии ассимилируют содержащиеся в барде питательные вещества, вырабатывая биогаз (смесь метана и углекислоты). Биогаз может быть использован в качестве котельного топлива, а накапливающийся осадок "--- как добавка к кормам и удобрение. 
Достоинством данного метода переработки является относительные низкие эксплуатационные затраты. К сожалению, данный способ ограничен по возможностям переработки концентрированных сред, отчего возникает необходимость в дополнительном разбавлении барды. 
Как следствие, для реализации используются метантанки объёмом порядка 2000 м$^3$, а значит значительные земельные участки), так как процесс переработки барды анаэробными бактериями недостаточно интенсивен. 
Другим недостатком метода является весьма длительный период выхода на режим "--- до 6 месяцев. 
Наконец, эксплуатация аппаратуры с горючими газами требует не только серьезных проектных согласований, но и кадров с соответствующей профессиональной подготовкой \cite{web_spbarda}. 
Большинству требований, выдвигаемых производителями спирта, будет удовлетворять золотая середина: что-то взять от имеющихся технологий переработки барды, что-то от очистки сточных вод. 
Наиболее разумным с учётом выше- перечисленных ограничений, следует признать путь по предлагаемой схеме (рис.~\ref{ShemaKomb}) (из расчета на 100 т исходной жидкой барды), по данным \cite{Novikov_2007_2_Journal}. 

\begin{figure} 
\centering 
\begin{footnotesize} 
\def\svgwidth{0.884\linewidth} 
\input{figures/ShemaKomb.pdf_tex} 
\end{footnotesize} 
\caption{Комбинированная схема утилизации послеспиртовой барды} 
\label{ShemaKomb} 
\end{figure} 

\begin{itemize} 
\item Механическими способами разделяют барду на влажный осадок нерастворимых веществ (дробину) и фильтрат с растворенными в нём веществами. 
С этим хорошо справляются сепараторы и центрифуги. 
Получают 10 т дробины с влажностью 65\% и 90~т горячего фильтрата с содержанием растворенных в нём сухих веществ, солей и кислот около 4\%. 
\item Дробину сушат простыми и надежными способами, желательно без дополнительных затрат пара, так как не все котельные осилят дополнительную нагрузку. 
Есть соответствующее оборудование, так что удалить 6 т влаги (6\% к общей исходной массе) вполне реально. Процесс будет более легким, если дробину предварительно смешать с отшелушенными сухими зерновыми оболочками, которые часть влаги <<оттянут>> на себя за счёт своей капиллярно-пористой структуры. Эту влагу легче извлечь с общей увеличенной поверхности при сниженной относительной влажности. 
\item Жидкий фильтрат разделяют мембранными (механогидравлическими) способами на 25\%-й сметанообразный коричневый концентрат растворенных веществ (16 т) и чистую горячую воду (74 т). 
Современные двухступенчатые установки с керамическими мембранами позволяют надежно выполнить такое разделение \cite{web_fermenter} даже при при высоких температурах фильтрата "--- до 100$\celcius$ включительно.
Существуют методики, способные повысить процент разделения путём получения многослойного фильтрующего слоя \cite{Makushin_2006_autoref} с добавлением в исходную барду специальных присадок. Солесодержание в воде в два раза ниже, чем в водопроводной, поэтому её выгодно вернуть в самое начало процесса производства спирта. 
Эта технология пока не стала массовой, однако, иного пути нет, так как механическое отделение влаги в 20 с лишним раз дешевле по текущим затратам, чем испарение при сушке и в 5 раз дешевле упаривания в выпарных установках. 
\item Белковый концентрат также подлежит сушке на ином типе сушильного устройства с более высокой напряженностью пространства сушилки по влаге. 
Удаление влаги сушкой в объёме около 12 т (12\% к общей исходной массе) "--- тоже сложный процесс, но размеры оборудования и энергозатраты будут на порядок меньше, чем это предлагается сейчас. 
\end{itemize} 

\textit{Итог}: на выходе получаем два сухих продукта с разной биологической ценностью и разной стоимостью. 
Так же существует множество разных способов модернизации непосредственно спиртового производства, такие как брожение в условиях вакуума \cite{web_npk,Arseniev_2001_4_Journal}, обработка исходного сырья ультразвуком \cite{Smirnova_2006_1_Journal}, применение инфракрасной микронизации \cite{Zverev_2006}. 
Первое заслуживает особого внимания. 
Совмещение процессов брожения и дистилляции под вакуумом позволяет исключить из состава технологической схемы бражную колонну. 
Совмещение брожения и дистилляции даёт также возможность исключения операций декантирования и выпаривания послеспиртовой барды. 
Спиртовой дистиллят направляется на ректификацию для получения биоэтанола, питьевого или технического спирта. 
Исключение из состава сушильного отделения деканторов и выпарной линии обеспечивается благодаря тому, что за счёт испарения воды концентрация сухих веществ в послеспиртовой барде в бродильных чанах к окончанию процесса брожения возрастает до уровня 26--30\%. 
А это позволяет направлять послеспиртовую барду непосредственно на сушилку. 
Улучшение качества получающейся сухой послеспиртовой барды обеспечивается благодаря следующим причинам. 
В процессе переработки зерна на спирт и сухие кормопродукты, ни само зерно, ни полупродукты на его основе не подвергаются воздействию высоких температур. 
Благодаря этому, содержащиеся в зерне витамины не разрушаются, а белки переходят в кормопродукты неденатурированными и практически полностью усваиваются сельскохозяйственными животными. 
В связи с особенностями технологии, к концу брожения за счёт испарения воды и роста биомассы дрожжей, содержание дрожжей в бражке составляет порядка 250~млн~кл/мл, а при классических технологиях 100--120~млн~кл/мл. 
В результате, в конце цикла брожения за счёт испарения воды, в бродильных чанах остается сметанообразная, сохраняющая текучесть жидкость, содержащая 26--30\% сухих веществ. 
Эта жидкость содержит, в том числе, 34--40\% сырого протеина (в пересчёте на активное сухое вещество) и, по существу, является высококонцентрированной кормовой добавкой, пригодной для сушки на сушилке барабанного или другого типа.


