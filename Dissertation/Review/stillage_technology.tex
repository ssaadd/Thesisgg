\section{Технологические схемы переработки спиртовой барды}

Известен способ приготовления кормовой добавки (рисунок~\ref{fig:stillage_bumbar}) на основе послеспиртовой барды и сои~\cite{Bumbar.Sposob.2003}.
Послеспиртовую барду с содержанием СВ\(=\)6--8\% объема центробежным насосом 1 из бардяной ямы 2 перекачивают в чан-сборник барды 3.
\begin{figure}[htb]
\centering
\includegraphics[width=\textwidth]{figures/temp/bumbar.jpg}
\caption[Способ приготовления кормовой добавки на основе послеспиртовой барды и сои]{Способ приготовления кормовой добавки на основе послеспиртовой барды и сои: 1, 9~"--- центробежный насос, 2~"--- бардяная яма, 3~"--- чан-сборник барды, 4, 7~"--- шнековый фильтр-пресс, 5, 8~"--- сборник фильтрата барды, 6, 15, 19~"--- шнековый транспортер, 10~"--- выпарная установка, 11, 16~"--- смеситель, 12~"--- испаритель, 13~"--- сушилка, 14~"--- молотковая дробилка, 17, 18~"--- сборник, 20~"--- гранулятор.}\label{fig:stillage_bumbar}
\end{figure}

Из чана-сборника 3 барду самотеком направляют в шнековый фильтр-пресс 4.
Из фильтр-пресса 4 фильтрат барды до 50\% объема с концентрацией СВ\(=\)2--3\% направляют самотеком в сборник фильтрата барды 5 и в последующем центробежным насосом перекачивают на технологические нужды спиртового производства: на рассортировку помола крахмалосодержащего сырья и для разжижения бражки.
Оставшуюся барду с содержанием СВ\(=\)25--30\% шнековым транспортом 6 направляют на второй фильтр пресс 7. 
Из фильтр--пресса 7 фильтрат барды СВ\(=\)3\% самотеком поступает в сборник фильтрата барды 8 и центробежным насосом 9 направляют в выпарную установку 10.
Барду с содержанием СВ\(=\)35--40\% с фильтр--пресса 7 направляют в смеситель 11, сюда же одновременно подают соевый помол до 40\% к объему барды, мел, соль, конденсат из испарителя 12 и самотеком из выпарной установки 10 бардяной концентрат СВ\(=\)40--50\%.
В рубашку сушилки 13 падают пар под давлением 5--6 атм.
Температура в сушилке поддерживается 130--140\(\celcius\). 
Из роторно--дисковой сушилки кормовая смесь с содержанием СВ\(=\)90--92\% поступает в молотковую дробилку 14, испарившуюся влагу кормовой смеси из сушилки направляют в испаритель 12.
Выходящий из молотковой дробилки 14 помол перемещают шнековым транспортером 15 в смеситель 16, где его смешивают с минеральными добавками и витаминами, поступающими в смеситель 16 из сборников 17 и 18.
Перемешанную кормовую массу подают шнековым транспортером 19 в гранулятор 20.
С гранулятора готовую к скармливанию кормовую добавку затаривают в мешки и направляют на реализацию.

Технологическая линия производства белково-витаминного кормопродукта из послеспиртовой зерновой барды~\cite{Antipov.Technologycal.2005} представлена на~\cref{fig:stillage_antipov}.
\begin{figure}[htb]
\centering
\includegraphics[width=\textwidth]{figures/temp/antipov.jpg}
\caption[Технологическая линия производства белково-витаминного кормопродукта из послеспиртовой зерновой барды]{Технологическая линия производства белково-витаминного кормопродукта из послеспиртовой зерновой барды: 1~"--- промежуточная емкость, 2~"--- устройство для обезвоживания, 3, 8~"--- вентилятор, 4~"--- калорифер, 5~"--- вихревая сушилка, 6~"--- смеситель, 7~"--- циклон, 9~"--- гранулятор, 10~"--- вентиляционная камера, 11~"--- сборник готовой продукции, 12~"--- устройство для упаковки, 13~"--- испаритель, 14~"--- вакуумный насос, 15~"--- барометрический конденсатор, 16~"--- промежуточная емкость, 17~"--- устройство для концентрирования фильтрата.}\label{fig:stillage_antipov}
\end{figure}
Послеспирт\-овая зерновая барда из брагоперегонного аппарата по магистрали подвода послеспиртовой зерновой барды поступает в промежуточную емкость 1, где аккумулируется в необходимом объеме для бесперебойной работы вакуумного фильтра 2 типа БОП со сходящим полотном и системой непрерывной регенерации ткани.
После вакуумного фильтра 2 послеспиртовая барда разделяется на два потока, один из которых включает твердый продукт (дробина), а другой~"--- жидкий (фильтрат).
Дробина выглядит как рыхлая масса с влажностью от 40\% до 60\%. Фильтрат включает в себя 2\dots4\% сухих веществ, часть которых находится во взвешенном состоянии, а часть растворена.
Работу фильтра 2 обеспечивает вакуумный насос 14.
Для получения более глубокого вакуума используют испаритель 13 и барометрический конденсатор 15.
С полотна вакуумного фильтра 2 дробину при помощи средства межоперационной передачи подают в сушилку 5.
После сушилки 5 дробина имеет влажность 15\dots20\% и ее можно подвергать грануляции в устройстве 9.
Из гранулятора 9 продукт попадает в вентиляционную камеру 10, где обдувается горячим сухим воздухом, поступающим из калорифера 4, доводится до содержания влаги 6\dots10\%, после чего он поступает в сборник 11 готовой продукции, упаковывается на устройстве 12, например, в мешки или насыпается россыпью в запирающуюся тару и отвозится на склад или потребителю.
Для увеличения выхода продукта и очистки отработанного воздуха из устройств 5 и 10 последний проходит через воздухоочиститель циклонного типа~"--- циклон 7, в котором отделяются твердые частицы белково-витаминного продукта, поступающие в дальнейшем в гранулятор 9.
В атмосферу выбрасывается практически чистый воздух.

Способ приготовления сухой барды и установка для его осуществления~\cite{Ezhkov.Sposob.2005} изображена на~\cref{fig:stillage_ezhkov}.
Исходную послеспиртовую барду подают в камеру подготовки 6 по тракту загрузки 7.
Одновременно через тракт загрузки 7 в камеру 6 подают и требуемое количество сухого продукта.
Количество подаваемого сухого продукта регулируют в зависимости от состава и консистенции послеспиртовой барды, типа сушильной камеры, вида несущей поверхности и т.п.
\begin{figure}[htb]
\centering
\includegraphics[width=\textwidth]{figures/temp/ezhkov.jpg}
\caption[Способ приготовления сухой барды и установка для его осуществления]{Способ приготовления сухой барды и установка для его осуществления: 1~"--- сушильный аппарат, с входным 2 и выходным 3 участками, 4~"--- раздаточный узел, 5~"--- блок повышения содержания сухих веществ, 6~"--- камера подготовки исходной послеспиртовой барды, 7, 8~"--- тракт загрузки и выгрузки соответственно, 9~"--- накопительный бункер, 10~"--- режущий скребок, 11~"--- рециркуляционная линия, 12~"--- несущая поверхность.}\label{fig:stillage_ezhkov}
\end{figure}
Для широкого спектра использующейся послеспиртовой барды оптимальным значением количества сухого продукта является 40--60\% от массового расхода исходной барды.
Предварительно, перед подачей в камеру подготовки, сухой продукт измельчают до размера частиц не более 1 мм.
Послеспиртовую барду и сухой продукт смешивают в камере 6 до образования однородной смеси, после чего образованную смесь с заданным содержанием сухих веществ в ней выводят из камеры 6 по тракту выгрузки 8.
С помощью подключенного к тракту выгрузки 8 раздаточного узла 4 на несущей поверхности 12 формируют сплошной слой барды не более 2 мм, и сформированная таким образом барда поступает на сушку в сушильный аппарат 1.
Перед операцией формования барды несущую поверхность смачивают веществом, предотвращающим прилипание барды к поверхности, например пищевым растительным маслом.
Т.к. образованная смесь, по существу, на 50\% состоит из частиц сухого продукта, смоченного исходной жидкой послеспиртовой бардой, процесс сушки такой смеси сводится к высушиванию внешней поверхности упомянутых частиц и поэтому значительно интенсифицируется.
После высушивания до требуемой кондиции высушенная барда подается к накопительному бункеру 9, где отделяется от несущей поверхности 12 с помощью режущего скребка 10, и полученный сухой продукт частично возвращается по рециркуляционной линии 11 в камеру подготовки 6, а остальной сухой продукт выводится по технологическому назначению.

Схема технологии получения сухой барды стандарта DDGS из отходов спиртового производства~\cite{CH.Device.2006} приведена на~\cref{fig:stillage_chinese}.
\begin{figure}[htb]
\centering
\includegraphics[width=\textwidth]{figures/temp/chinese.jpg}
\caption[Технология получения DDGS продукта из отходов спиртового производства]{Технология получения DDGS продукта из отходов спиртового производства:
1~"--- сепаратор, 2~"--- выпарная установка, 4~"--- смеситель, 5~"--- сушилка: 5--1~"--- парогенератор, 5--3~"--- первая вихревая сушилка, 5--4, 5--7~"--- циклон, 5--5~"--- генератор горячего воздуха, 5--6~"--- вторая вихревая сушилка, 5--8~"--- вентилятор.}\label{fig:stillage_chinese}
\end{figure}
Барду из цеха подают в сепаратор 1, далее фильтрат направляют в выпарной аппарат 2 и затем в смеситель 4, куда подают твердую фазу после сепаратора 1.
После смешивания продукт подают в сушилку 5, включающую в себя парогенератор 5--1, вихревые сушилки 5--3 и 5--6, циклоны 5--4 и 5--7.
Материал высушивается горячим воздухом от генератора горячего воздуха 5--5 подключенного к обоим сушилкам 5--3 и 5--6. 
Испарившийся из материала пар после циклона 5--4, направляется в выпарной аппарат 2, смешиваясь в соединительном трубопроводе циркуляционным вентилятором 5--2 с перегретым паром после генератора 5--1.
Вентилятор 5--8 после циклона 5--7 выбрасывает отработанный пар через выпускной канал. 
Выпарная установка 2 снабжена циркуляционным насосом 3.

Известен способ производства сухой гранулированной барды стандарта DDGS \cite{Meier.Method.2008} (рисунок~\ref{fig:stillage_meier}).
\begin{figure}[htb]
\centering
\includegraphics[width=\textwidth]{figures/temp/meier.jpg}
\caption[Способ производства сухой барды стандарта DDGS]{Способ производства сухой гранулированной барды стандарта DDGS:
22~"--- смеситель, 23~"--- гранулятор, 28~"--- насос, 29~"--- спускной патрубок, 30~"--- СВЧ-сушилка, 31, 36~"--- транспортер, 32~"--- циклон, 33~"--- вентилятор.}\label{fig:stillage_meier}
\end{figure}
DWG из центрифуги подается в смеситель 22, в который задают CDS, ферменты и/или другие добавки.
Далее, смесь DWG гранулируют с образованием гранул 27.
Гранулятор 23 оказывает давление на смесь DWG и приводит его в состояние агрегации и/или агломерации материала с соответствующим увеличением плотности продукта. 
Гранулятор 23 может работает под вакуумом с использованием вакуумного насоса 28.
Гранулятор 23 оборудован сливным патрубком 29, чтобы обеспечить выход избытка воды.
Кроме того, CDS и/или ферменты могут быть добавлены к DWG в гранулятор 23 таким же образом, как в смеситель 22.
После гранулятора 23 смесь DWG в виде гранул, по конвейеру 31 подают в СВЧ-сушилку 30.
Для сушки используют вентилятор 33 он обеспечить движущую силу потока воздуха через сушилку 30 и циклон 32.
Очищенный после циклона 32, материал возвращается в смеситель 22 транспортером 36.

