\section{Общая характеристика зерновой барды}

Барда представляет собой сложную полидисперсную систему, сухие вещества в которой находятся в виде взвесей или в растворенном состоянии.
При переработке на спирт крахмалистого сырья в барду переходят сухие вещества бражки, за исключением углеводов, из которых образуются спирт, диоксид углерода и другие летучие продукты.
Зерно-картофельная барда содержит до 92\% воды, 8\% сухих веществ и имеет кислую реакцию (pH~4,2--4,6).
Сухие вещества барды состоят из белков, гемицеллюлоз, целлюлозы, сахаров, декстринов, жира, минеральных и других веществ.
Наличие в ней значительного количества легкоусвояемых белковых и безазотистых веществ, витаминов делает зерно-картофельную барду ценным кормовым продуктом.
Сухие вещества барды состоят на 35--40\% из взвешенных веществ и на 55--65\% -- из растворимых.
Относительная плотность барды колеблется от 1,02 до 1,08 и в среднем составляет 1,04.
Выход барды зависит от содержания спирта.
В таблице \ref{tab:HimSostav} указан химический состав барды из различных культур.

\LTXtable{\textwidth}{tabs/chemical_sostav}

В таблице \ref{tab:Pitatel} показаны химические составы послеспиртовой барды, отличающиеся между собой из-за разных видов используемого сырья и технологий получения спирта.

\LTXtable{\textwidth}{tabs/chemical_pitatelnost}

По химическому составу и питательности сухая барда схожа с белотином и биотрином, но содержит повышенный уровень клетчатки.
В ней меньше лизина, но больше метионина и цистеина.

В процессе переработки барды на производстве не меняется кислотность барды, фильтрата и сиропа.
Она составляет $PH = 4-4.2$.
Концентрация сухих веществ в исходной барде примерно 7--10\%.
Концентрация сухих нерастворимых веществ в фильтрате 0.2--1\%.
Концентрация сухих растворимых веществ в фильтрате от 2 до 4\%.
Концентрация сухих веществ в сиропе (жидкость выходящая с третьего корпуса выпарной установки) примерно 28--30\%.
Размер отдельных частиц мелкодисперсной фракции фильтрата можно оценить в районе 0.007--0.008 мм, а отдельные агломераты этих частиц в размере до 0.05 мм.
В сиропе также как и в фильтрате размер частиц твердой мелкодисперсной фазы практически такой же.
Разница состоит только в концентрации твердой фазы. 
\newline Размер отдельных частиц мелкодисперсной фракции сиропа можно оценить в районе 0.007--0.008 мм.
А отдельные агломераты этих частиц в размере 0.05 мм.~\cite[с.~68]{Pahomov.Analiz.2013}
Наблюдаются крупные частицы размером до 0.03--0.05 мм.
В основном мелкодисперсная фаза имеет ориентировочный размер 0.01 мм.
Иногда встречаются некоторые упорядоченные (агрегатированные) частицы размером до 0.3 мм, состоящие из отдельных частиц размером от 0.03 до 0.05 мм.
Иногда встречаются крупные включения "--- частицы неправильной угловатой формы размером от 0.5 до 1 мм.
Также наблюдаются отдельные крупные тонкие пластинчатые включения "--- частицы почти прямоугольной формы размером от 0.3 до 0.5 мм и включения в виде длинных <<нитей>> толщиной около 0.02 "--- 0.03 мм.
Микроскопические исследования жидкой барды показали, что этот дисперсный материал можно отнести к тонким суспензиям с включениями средне и крупно дисперсных частиц.~\cite[с.~67]{Pahomov.Analiz.2013}

\subsection{Состав твердой и жидкой фазы барды}

Анализ твёрдой фазы барды проводился после её отделения на микрофильтрах и двойной промывки осадка обессоленной водой при объёме порции воды, равном объёму фильтрата.
Твёрдая фаза представляет собой непрогидролизованные остатки дробленого зерна и выросшую на стадии спиртового брожения дрожжевую биомассу.
Дисперсный анализ дробины показал, что размер и масса частиц имеют очень широкий разброс: при среднем размере 0,5~мм диапазон составляет 0,0312~мм.
Химический состав дробины (таблица \ref{Drobina}) зависит от многих технологических факторов, в первую очередь от состава и качества исходного сырья, а также от режимов механической, тепловой и ферментативной деструкции крахмала и белков.
Стадии спиртового брожения и отгонки спирта не вносят существенных изменений в состав дробины.
Собранная на микрофильтре дробина представляет собой плотную массу однородной консистенции от темно-желтого до коричневого цвета.
Перед проведением анализов влажный осадок высушивали до остаточной влажности 12\%, после чего сухая дробина может храниться неограниченно долго.

\LTXtable{\textwidth}{tabs/drobina_sostav}

Отсюда видно, что твёрдая фаза барды не является биологически ценным кормовым продуктом по причине низкого содержания белка. Тем не менее, благодаря сохранению большей части белка в составе твёрдой фазы произошло обогащение её белком из-за утилизации крахмала.
В таблице \ref{tab:=Amin} представлены результаты исследования аминокислотного состава белка в исходном пшеничном зерне и в дробине.

\LTXtable{\textwidth}{tabs/aminoacid_sostav}

Микрофильтрат барды представляет собой прозрачную жидкость светло-коричневого цвета.
Общее содержание органических веществ в ней может оцениваться по стандартным показателям ХПК (химическое потребление кислорода) и БПК (биологическое потребление кислорода).
Таким образом, фильтрат барды является источником большого количества разнообразных органических веществ, при небольшом содержании минеральных.
Некоторое повышение концентрации минеральных веществ по сравнению с их содержанием в свежей воде (0,5~г/л) объясняется добавлением питательных солей на стадии дрожжеращения и спиртового брожения.
Жиры частично растворились в жидкой фазе на стадии разваривания и транзитном перешли в барду.
Показатель <<сырой протеин>> объединяет в себе пептиды и аминокислоты и некоторое количество водорастворимых белков. Пептиды и аминокислоты образуются в основном на стадии осахаривания как продукты гидролиза белков.
Органические кислоты представлены в основном следующими низкомолекулярными соединениями: уксусная кислота, масляная и изомасляная, валериановая и изовалериановая кислоты, муравьиная и изопропионовая кислоты.
Они появляются в жидкой фазе на стадии спиртового брожения как продукты метаболизма микроорганизмов.
Углеводы (неразложившийся крахмал и неутилизированные сахара) являются признаком несоблюдения технологического режима основного производства и обычно в барде отсутствуют.
В фильтрате зерновой барды методом хроматографии определено наличие 12 аминокислот (в\% к содержанию белка): аланин 9,8 аргинин 6,8 аспарагиновая кислота 3,2 глютаминовая кислота 1,1 изолейцин 6,2 лейцин 6,0 лизин 7,6 метионин 3,4 фенилаланин 5,6 треонин 7,8 тирозин 4,8 валин 1,7.
Всего 64.

Спиртовая дробина имеет высокую начальную влажность (более 250\% на с.в.), что не позволяет ее длительно хранить.
Поэтому целесообразно ее обезвоживать.
По своему составу спиртовая дробина близка к пивной дробине и барде спиртовой.
Сходным сырьем в исследованиях была пшеница.
Сравнительный состав продуктов представлен в табл. , откуда видно, что спиртовая дробина по таким показателям, как сырой протеин и сырой жир превосходит пшеницу, дробину пивную и барду спиртовую.~\cite{Oleinikov.Svoystva.2010}


По результатам испытаний, проведенных в ОАО АХЦ <<Удмуртский>> (2009 г.) фугат послеспиртовой барды содержит: сухого вещества "--- 8.29\%, азота "--- 0.34\% , фосфора "--- 0.11\% , калия "--- 0.03\% , кальция "--- 0.01\%.
Концентрация в фугате барды токсичных элементов и радионуклидов, остаточных количеств пестицидов соответствует нормативным требованиям.
Фугат послеспиртовой барды не обладает фитотоксичностью при использовании в дозе до 300 т/га, усиливает азотминирализационную эффективностью почвы на 0.19 мгN/кг при внесении 1 т агрохимиката.~\cite{Makarov.Ocenka.2010}


\subsection{Сравнительный анализ}


Cпиртовая дробина является ценным продуктом, содержащим незаменимые аминокислоты и высокую питательную ценность~\cite{Oleinikov.Svoystva.2010}



\LTXtable{\textwidth}{tabs/sravnenie_analiz_barda}

\LTXtable{\textwidth}{tabs/drobina}

\LTXtable{\textwidth}{tabs/aminoacid_sravnenie}

Y.~Kim и\,др провели сравнительный анализ полупродуктов переработки послеспиртовой барды.
Сухая барда и исходная барда богаты глюканом, ксиланом и арабмнаном, источниками сбраживаемых сахаров при производстве спирта.
Общее содержание сахаров (глюкан и ксилан) сухой барды и исходной барды 29,4\% и 33,4\% соответственно, в пересчете на сухое вещество.
Сырой протеин составляет 25\% к сухому веществу сухой барды.
Сырой жир составляет 11,6\%~\cite[pp.~5171--5172]{Kim.Composition.2008}

\LTXtable{\textwidth}{tabs/sravnenie_product_for_ddgs}


\subsection{Анализ готовой сухой барды}


Зарубежными исследователями проведен обзорный анализ многочисленных работ на тему химического состава DDGS~\cite{Liu.Chemical.2011}.

\LTXtable{\textwidth}{tabs/DDGS_analiz}

\LTXtable{\textwidth}{tabs/DDGS_mineral}


