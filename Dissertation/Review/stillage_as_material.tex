\section{Барда как объект производства}

Химический состав исходного сырья для производства спирта, состав барды из крахмалсодержащего сырья и фракционный состав белка барды представлены в~\cref{tab:grain_chem,tab:stillage_sostav,tab:fraction_sostav}~\cite{Zueva.Fraction.2013}.

\LTXtable{\textwidth}{tabs/grain_chem_sostav}

\LTXtable{\textwidth}{tabs/stillage_sostav}

\LTXtable{\textwidth}{tabs/fraction_sostav}

Сравнительный анализ \cite{Kim.Composition.2008} состава DDGS, фильтрата барды и сухой барды приведен в~\cref{tab:ddgs_stillage_thinstillage}.
Сухая барда и исходная барда богаты глюканом, ксиланом и арабинаном, источниками сбраживаемых сахаров при производстве спирта.
Общее содержание сахаров (глюкан и ксилан) сухой барды и исходной барды 29,4\% и 33,4\% соответственно, в пересчете на сухое вещество.
Сырой протеин составляет 25\% к сухому веществу сухой барды.
Сырой жир составляет 11,6\%
 
\LTXtable{\textwidth}{tabs/sravnenie_product_for_ddgs}

Cпиртовая дробина является ценным продуктом, содержащим незаменимые аминокислоты и имеющая высокую питательную ценность.~\cite[с.~158]{Oleinikov.Svoystva.2010}.
Химический состав различных, сходных по показателям с бардой, продуктов приведены в \cref{tab:chem_sostav_feed_products}, аминокислотный состав "--- в~\cref{tab:aminoacid_sostav_cake_stillage}.%~\cite{Oleinikov.Svoystva.2010}

\LTXtable{\textwidth}{tabs/sravnenie_analiz_barda}

%\LTXtable{\textwidth}{tabs/drobina}

\LTXtable{\textwidth}{tabs/aminoacid_sravnenie}

В~\cite{Liu.Chemical.2011} приведен обзор многочисленных зарубежных источников по анализу состава барды.
Данные сведены в таблицу~\ref{tab:DDGS_analiz}. Минеральный состав приведен в \cref{tab:DDGS_mineral}.

\LTXtable{\textwidth}{tabs/DDGS_analiz}

\LTXtable{\textwidth}{tabs/DDGS_mineral}

Образцы DDGS по~\cite[p.~594]{Rosentrater.Some.2006} имели следующие теплофизические показатели: среднее содержание влаги \(14,7\)\%, активность воды \(0,55\), теплопроводность \(0,07\) \(\text{Вт}/\text{м}\celcius\), сопротивление \(14,0\) \(\text{м}\celcius/\text{Вт}\), коэффициент диффузии \(0,13\) \(\text{мм}^2/\text{с}\), объемная плотность \(483,3\) \(\text{кг}/\text{м}^3\), угол естественного откоса \(31,5^\text{o}\).
Таким образом, свойства барды аналогичны другим сухим кормовым ингредиентам, таких как кукурузный глютен, корма и другим ингредиенты на основе кукурузы.
Теплофизические свойства спиртовой барды на ряде предприятий по~\cite{Rosentrater.Some.2006} приведены в~\cref{tab:stillage_physical}.

\LTXtable{\textwidth}{tabs/stillage_physical}

