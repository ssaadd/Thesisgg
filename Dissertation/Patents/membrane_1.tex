\subsection{Мембранный аппарат}

Мембранный аппарат (рисунок~\ref{fig:patent_membrane_ap}) представляет собой корпус 1, выполненный из
непроницаемого материала, с патрубками для ввода исходного раствора 8, вывода фильтрата 5 и концентрата 4, с расположенным внутри него трубчатым мембранным модулем 2 конусной формы с нанесенной на его внутреннюю поверхность полупроницаемой мембраной 12.
Турбулизатор 6 выполнен в виде конусообразного вала с винтовыми спиралями и состоит из трех участков: первый участок выполнен в виде ступицы, установленной в подшипник 7 с возможностью осевого перемещения в подводящем патрубке исходного раствора 8, на конце которого смонтирован пропеллер 14 с лопастями, вращающимися под действием входного потока жидкости, и передачей крутящего момента турбулизатору 6.
Второй участок турбулизатора 6, находящийся в мембранном модуле, выполнен в виде конусообразного вала с винтовыми спиралями, вращение которого обеспечивает перенос исходного раствора вдоль мембранного модуля 2, при этом турбулизатор 6 совершает возвратно-поступательное движение путем принудительного изменения давления исходного раствора в подводящем патрубке исходной жидкости 8. 
Третий участок турбулизатора 6 выполнен в виде цилиндра и установлен в подшипнике 17, который закреплен в кожухе 13, с возможностью ограничения возвратно-поступательного движения от действия пружины сжатия 3, установленной в стакане 10 со стороны отвода концентрата.
Для обеспечения вращения турбулизатора 6 поток жидкости приводит в движение лопасти пропеллера 14, установленного для передачи крутящего момента валу турбулизатора 6 в подводящем патрубке исходного раствора 8.
Мембранный модуль 2 коаксиально расположен в корпусе 1, выполненном из непроницаемого материала.
Неподвижный ембранный модуль 2 и корпус 1 герметично соединены между собой при помощи фланца 15 и глухого фланца 16.
На торцевой части корпуса 1 с одной стороны установлен подводящий патрубок исходного раствора 8, с другой "--- патрубок вывода фильтрата 5.
Корпус 1 снабжен патрубком вывода концентрата 4 из пространства, образованного наружной поверхностью неподвижного мембранного модуля 2 и внутренней поверхностью корпуса 1.
Фланец 15 выполнен с отверстием для размещения прокладки 11.

Исходный раствор подается с помощью подводящего патрубка исходного раствора 8 в мембранный модуль 2, где пропеллер 14, закрепленный на первом участке турбулизатора 6, приводит его в постоянное вращение, за счет раскручивания лопастей пропеллера 14 путем непрерывной подачи исходного раствора. 
Вместе с этим, турбулизатор 6 совершает возвратно-поступательное движение вдоль мембранного модуля 2 за счет винтовых спиралей на поверхности турбулизатора 6, а также установленной с обратной стороны турбулизатора 6 пружины сжатия 3.
Перенос исходной жидкости вдоль мембранного модуля 2 осуществляется витками турбулизатора 6 при этом фильтрат, прошедший через полупроницаемую мембрану 12 нанесенную на неподвижный мембранный модуль 2, поступает в полость, образованную внешней стенкой мембранного модуля 2 и внутренней поверхностью корпуса 1, откуда он отводится при помощи патрубка вывода фильтрата 5. 
При снижении производительности аппарата в ходе процесса разделения, из-за снижения проницаемости полупроницаемой мембраны 12 новый поток исходного раствора подают с увеличенным давлением, из-за чего пружина сжатия 3, установленная в стакане 10 со стороны отвода концентрата, который закреплен в кожухе 13 приводит турбулизатор 6 в возвратно-поступательное движение. 

\begin{figure}[!htb]
\centering
\begin{small}
\begin{subfigure}[htb]{\linewidth}
\centering
\def\svgwidth{12cm}
\input{figures/membrane1_fig1.pdf_tex}
\caption{}
\label{fig:main_view_membrane1}
\end{subfigure}
\quad
\begin{subfigure}[htb]{0.45\linewidth}
\centering
\def\svgwidth{7cm}
\input{figures/membrane1_fig2.pdf_tex}
\caption{}
\label{fig:main_view_membrane2}
\end{subfigure}
\quad
\begin{subfigure}[htb]{0.45\linewidth}
\centering
\def\svgwidth{7cm}
\input{figures/membrane1_fig3.pdf_tex}
\caption{}
\label{fig:main_view_membrane3}
\end{subfigure}
\end{small}
\caption[Мембранный аппарат]{Мембранный аппарат: (а) "--- основной вид аппарата; (б) "--- вид сбоку; (в) "--- торцевая часть пружинного механизма (увеличено).
Корпус 1, трубчатый мембранный модуль 2, пружина сжатия 3, патрубки: вывода концентрата 4 и фильтрата 5, ввода исходного раствора 8, турбулизатор 6, подшипники 7, 17, стакан 10, прокладка 11, полупроницаемая мембрана 12, кожух 13, пропеллер 14, фланцы 15, 16.}
\label{fig:patent_membrane_ap}
\end{figure}


При этом происходит удаление с поверхности полупроницаемой мембраны 12 предыдущего слоя концентрата и восстановлению селективности и проницаемости полупроницаемой мембраны 12, при этом витки вращающегося турбулизатора 6, удаляют концентрат через патрубок вывода концентрата 4.
После этого все процессы повторяются аналогично описанным выше.
Предложенный мембранный аппарат позволяет обеспечить:

низкий уровень концентрационной поляризации на поверхности полупроницаемой мембраны за счет постоянного изменения во времени гидродинамического режима в мембранном канале трубчатого мембранного модуля;

многозадачный режим работы мембранного аппарата для разделения, концентрирования и опреснения различных растворов методами обратного осмоса и ультрафильтрации, например, при вращении турбулизатора, при возвратно-поступательном перемещении турбулизатора, при одновременном вращении и возвратно-поступательном перемещении турбулизатора;

широкий диапазон производительности мембранного аппарата за счет различных вариаций турбулизатора, перенастраиваемой жесткости пружин;

абсолютную сохранность мембран по причине отсутствия непосредственного контакта турбулизатора с их поверхностью.

