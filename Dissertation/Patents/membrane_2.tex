\subsection{Вертикальный мембранный аппарат}

\begin{figure}[htb]
\centering
\begin{small}
\begin{subfigure}[htb]{\linewidth}
\centering
\def\svgwidth{10.5cm}
\input{figures/membrane2_fig1.pdf_tex}
\caption{}
\label{fig:main_view_vertical_membrane1}
\end{subfigure}
\begin{subfigure}[htb]{0.45\linewidth}
\centering
\def\svgwidth{7cm}
\input{figures/membrane2_fig2.pdf_tex}
\caption{}
\label{fig:main_view_vertical_membrane2}
\end{subfigure}
\begin{subfigure}[htb]{0.45\linewidth}
\centering
\def\svgwidth{7cm}
\input{figures/membrane2_fig3.pdf_tex}
\caption{}
\label{fig:main_view_vertical_membrane3}
\end{subfigure}
\end{small}
\caption[Вертикальный мембранный аппарат]{Вертикальный мембранный аппарат: (а) "--- основной вид аппарата; (б) "--- увеличенная торцевая часть аппарата; (в) "--- сечение планетарной передачи. Корпус 1, патрубки: ввода исходного раствора 5, вывода фильтрата 19, мембранный модуль 2, турбулизатор 3, подшипники 4, 14, 17, 23, спицы 6, прокладка 7, 16, полупроницаемая мембрана 8, пропеллер 9, фланец 10, 11, крышка 12, ведомые водила 13, крышка подшипника 15, ведущее центральное зубчатое колесо 18, отбойник 20, сателлиты 21.}
\label{fig:patent_vertical_membrane_ap}
\end{figure}

Вертикальный мембранный аппарат изображен на~\cref{fig:patent_vertical_membrane_ap} представляет собой корпус 1, выполненный из непроницаемого материала, с патрубками для ввода исходного раствора 5, вывода фильтрата 19, с соосно расположенным внутри него мембранным модулем 2 конусной формы с нанесенной на его внутреннюю поверхность полупроницаемой мембраной 8.
Турбулизатор 3 выполнен в виде конусообразного вала с винтовыми спиралями и состоит из трех участков: первый участок выполнен в виде ступицы, закрепленной спицами 6 и установленной в подшипник 4, в патрубке для ввода исходного раствора 5, на конце турбулизатора смонтирован пропеллер 9 с лопастями, вращающимися под действием входного потока жидкости с передачей крутящего момента турбулизатору 3.
Второй участок турбулизатора 3, находящийся в мембранном модуле 2, выполнен в виде конусообразного вала с винтовыми спиралями, вращение которого обеспечивает перенос исходного раствора вдоль мембранного модуля 2.
При этом турбулизатор 3 совершает вращательное движение путем принудительного изменения давления исходной жидкости в патрубке для ввода исходной жидкости 5 посредством раскручивания лопастей пропеллера 9.
Третий участок турбулизатора 3, выполненный в виде полого цилиндра, снабженного окнами для удаления концентрата, установлен в подшипнике 17 и закреплен в мембранном модуле 2 с крышкой подшипника 15.
Расположенная со стороны отвода концентрата планетарная зубчатая передача состоит из ведущего центрального колеса 18, ведомых водил 13 и трех сателлитов 21, вращающихся вместе с каждым из водил 13 вокруг центральной оси турбулизатора 3.
Причем в каждом из водил 13, жестко закрепленных в мембранном модуле 2, установлены подшипники 14.
Для защиты планетарной передачи от попадания на ее поверхность фильтрата прошедшего через полупроницаемую мембрану 8, к мембранному модулю жестко закреплен отбойник 20.
Мембранный модуль 2 и корпус 1 герметично соединены между собой при помощи фланца 10, прокладки 7, крышки 12 и фланца 11.
На торцевой части корпуса 1 с одной стороны установлен патрубок ввода исходного раствора 5, с другой, во фланце 11, "--- патрубок вывода фильтрата 19.
Между фланцем 10 и подшипником 23 установлена прокладка 16.

Исходный раствор подается через патрубок для ввода исходного раствора 5 в мембранный модуль 2, и под действием потока приводит во вращение пропеллер 9, закрепленный на первом участке турбулизатора 3, а, следовательно, и турбулизатор 3. 

Винтовые спирали на поверхности турбулизатора 3 позволяют увеличить крутящий момент и обеспечить необходимую скорость вращения, что способствует движению исходной жидкости вдоль мембранного модуля 2. 
Фильтрат, прошедший через полупроницаемую мембрану 8, нанесенную на мембранный модуль 2, поступает в полость, образованную внешней стенкой мембранного модуля 2 и внутренней поверхностью корпуса 1, откуда он отводится через патрубок вывода фильтрата 19. 
Раскручивание турбулизатора 3 приводит во вращение каждое из водил 13, закрепленных в мембранном модуле 2, которые вместе с центральным колесом 18 и тремя сателлитами 21 образуют планетарную зубчатую передачу. 
Она обеспечивает вращение мембранного модуля 2, что создает дополнительные условия для удаления с поверхности полупроницаемой мембраны 9 предыдущего слоя концентрата и восстановлению без дополнительных затрат селективности и проницаемости полупроницаемой мембраны 8. 
Витки вращающегося турбулизатора 3, удаляют концентрат через полую часть турбулизатора 3 и позволяют в значительной степени обеспечить отрыв молекул фильтрата от поверхности фильтрации. 
Таким образом предложенный вертикальный мембранный аппарат имеет следующие преимущества: 

 


создает благоприятные условия для снижения поляризационной концентрации за счет вращения мембранного модуля при котором достигается максимально возможный отрыв молекул фильтрата от поверхности фильтрации; 

снижает давление жидкости на входе в аппарат, а, следовательно, и энергозатраты на прокачку жидкости через аппарат на 5--10\%; 

сохраняет мембрану от механического повреждения по причине отсутствия непосредственного контакта турбулизатора с её поверхностью; 

обеспечивает многозадачный режим работы мембранного аппарата для разделения, концентрирования и опреснения различных растворов методами обратного осмоса и ультрафильтрации.

