\section{Способ получения порошкообразного продукта из фильтрата послеспиртовой барды}

Предложен способ получения порошкообразного продукта из фильтрата спиртовой барды, он предусматривает грубое и тонкое разделение барды в двух установленных параллельно сепараторах и фильтрах тонкой очистки, каждый из которых периодически работает в режиме разделения с отводом кека и получением фильтрата и в режиме противоточной водной регенерации фильтрующих элементов.
Выпаривание полученного после фильтра тонкой очистки фильтрата с концентрацией сухих веществ 4--5\% в вакуум-выпарном аппарате под разрежением 0,3--0,5~ атм и температуре кипения 70--80~\(\celcius\) с получением сгущенного раствора с концентрацией сухих веществ 30--40\%, из которого в распылительной сушилке получают порошкообразный продукт с влажностью 8--10\% и дисперсностью 70--80~мкм.
Разрежение в вакуум-аппарате создают посредством пароэжекторной установки, включающей парогенератор, эжектор, конденсатор, сборник конденсата и насос, которые работают в замкнутом термодинамическом цикле, причем одну часть рабочего пара с давлением 1--1,2~атм из парогенератора подают в греющую камеру вакуум-выпарного аппарата, а другую "--- с давлением 3--3,2~атм направляют в сопло эжектора.
Смесь паров "--- отработанного рабочего и эжектируемого из вакуум-выпарного аппарата с температурой 110--120~\(\celcius\) направляют в конденсатор для подогрева воздуха до температуры 75--80~\(\celcius\) с последующей подачей его в распылительную сушилку.
Далее отработанный воздух отводят в теплообменник-рекуператор, где снижают его температуру до точки росы и конденсируют содержащуюся в нем капельную жидкость в количестве испаряемой из продукта влаги, а затем осушенный воздух после подогрева в конденсаторе направляют в распылительную сушилку с образованием замкнутого цикла.
Конденсаты, полученные после теплообменника-рекуператора, греющей камеры вакуум-выпарного аппарата и из конденсатора, отводят в сборник конденсата, из которого одну часть конденсата насосом направляют в парогенератор для пополнения в нем уровня воды, а другую "--- подают в сепаратор и фильтр тонкой очистки, работающие в режиме противоточной водной регенерации.


Способ получения порошкообразного продукта из фильтрата спиртовой барды (\cref{fiq:patent_powder})
осуществляется следующим образом.

Исходную барду из аппаратного цеха спиртового завода подают в сепаратор 1, в котором осуществляют ее грубое разделение на кек и фильтрат. 
Далее фильтрат по линии 9.7 направляют на тонкое разделение в фильтр тонкой очистки 2, после которого кек соединяют с кеком, полученным после сепаратора 1, и отводят по линии 0.8, а фильтрат с концентрацией сухих веществ 4\dots5\%, подают в вакуум - выпарной аппарат 5. 
Фильтрат в аппарате 5 выпаривают под разряжением 0,3\dots0,5~атм. и получают сгущенный раствор с концентрацией сухих веществ 30\dots40\%, который с помощью вентиля 17 по линии 9.8 направляют в распылительную сушилку 14. 
На выходе из сушилки 14 получают порошкообразный продукт с влажностью 8\dots10\% и дисперсностью 70\dots80~мкм. 
Разряжение в аппарате 5 создают с помощью пароэжекторной установки, включающей парогенератор 12, эжектор 6, конденсатор 7, сборник конденсата 9 и насос 13, работающих в замкнутом термодинамическом цикле. 
Полученный в парогенераторе 12 рабочий пар с давлением 3\dots3,2~атм. разделяют на две части, одну из которых по линии 2.2 направляют в редукционный вентиль 8 для снижения давления до 1\dots1,2~атм., и далее в греющую камеру вакуум - выпарного аппарата 5, а другую часть рабочего пара с давлением 3\dots3,2~атм. в сопло эжектора 6. 

\begin{figure} 
\centering
\begin{small}
\def\svgwidth{\linewidth}
\input{figures/shemadrying.pdf_tex}
\end{small}
\caption[Способ получения порошкообразного продукта из фильтрата послеспиртовой барды]{Способ получения порошкообразного продукта из фильтрата послеспиртовой барды:
сепараторы 1, 3; фильтры тонкой очистки 2, 4; вакуум-выпарной
аппарат 5; эжектор 6; конденсатор 7; вентиль редукционный 8; сборник конденсата 9;
насосы 10, 13; вентиль предохранительный 11; парогенератор 12; распылительную
сушилку 14; теплообменник-рекуператор 15; вентилятор 16; вентиль 17; линии
материальных потоков: 0.1 "--- порошкообразный продукт; 0.8 "--- кек барды; 1.0 "---
отработанную воду; 1.6 "--- холодную воду; 1.8 "--- конденсат; 2.0 "--- пар отработанный; 2.2
"--- рабочий пар; 2.7 "--- смесь рабочего и отработанного пара; 3.0 "--- отработанный сушильный
агент; 3.3 "--- сушильный агент; 9.1 "--- исходную барду; 9.7 "--- фильтрат барды; 9.8 "--- сгущенный
фильтрат.}\label{fiq:patent_powder}
\end{figure}

Эжектируемые по линии 2.0 пары из вакуум-аппарата создают в нем разряжение 0,3\dots0,5~атм. при температуре кипения раствора $70\dots80\celcius$. 
Данная температура позволяет сохранить в растворе такие полезные вещества как: витамины (рибофлавин, тиамин, никотиновая кислота, пантотеновая кислота, биотин, холин), сырой протеин, углеводы. 
Смесь отработанного рабочего и эжектируемого паров с температурой $110\dots120\celcius$ по линии 2.7 направляют в конденсатор 7, в котором за счет рекуперативного теплообмена осуществляют подогрев воздуха до температуры $75\dots80\celcius$. Нагретый воздух вентилятором 16 по линии 3.3. подают в распылительную сушилку 14. 
Использование распылительной сушилки позволяет сократить продолжительность процесса сушки, которая может составлять от 15 до 30~с. 
При этом температура у частиц продукта в сушильной камере практически равна температуре испарения чистой влаги. 
Это связано с тем, что частицы имеют насыщенную поверхность. 
Сушка проходит практически мгновенно. 
В сочетании с невысокой температурой диспергируемых частиц продукта это позволяет получить высококачественный порошкообразный продукт. 
Такой метод сушки не вызывает денатурацию белков, окисления и потерь витаминов. 
Отработанный воздух из сушилки 14 по линии 3.0 отводят в теплообменник-рекуператор 15, куда по линии 1.6 подают холодную воду, а по линии 1.0 выводят отработанную воду.
За счет рекуперативного теплообмена между холодной водой и отработанным после сушилки воздухом происходит снижение его температуры до точки росы. 
При этом осуществляют конденсацию содержащейся в отработанном воздухе капельной жидкости в количестве испаряемой из продукта влаги на охлаждающей поверхности теплообменника-рекуператора. 
Осушенный таким образом воздух по линии 3.3 направляют сначала в конденсатор 7, а затем вновь в распылительную сушилку 14 с образованием замкнутого цикла. 
Образовавшийся конденсат после теплообменника-рекуператора 15 вместе с конденсатом после греющей камеры вакуум-выпарного аппарата 5 и конденсатора 7 по линиям 1.8 отводят в сборник конденсата 9. 
При этом одну часть конденсата из сборника 9 насосом 13 направляют в парогенератор 12, оснащенного предохранительным вентилем 11, для пополнения в нем уровня воды. 
Другую часть конденсата насосом 10 подают в сепаратор 3 и фильтр тонкой очистки 4, работающих в режиме противоточной водной регенерации для восстановления пропускной способности фильтрующих элементов. предлагаемый способ получения порошкообразного продукта из фильтрата спиртовой барды дает возможность:

повысить качество получаемого порошкообразного продукта за счет сохранения в нем полезных веществ и витаминов;

снизить энергозатраты за счет использования пароэжекторной установки и контура рециркуляции по сушильному агенту;

снизить энергозатраты на процессы выпаривания и сушки фильтрата вследствие использования теплоты отработанного пара после эжектора для нагрева воздуха перед подачей его в распылительную сушилку;

повысить эффективность разделения исходной барды на взвешенную и жидкую фракции и качество готового порошообразного продукта за счет использования распылительной сушилки, использующей в качестве сушильного агента воздух с невысокой температурой в $75\dots80\celcius$;

исключить загрязнение окружающей среды при проведении процесса сушки за счет наличия контура рециркуляции по сушильному агенту и обеспечить экологически безопасные условия эксплуатации оборудования.
%Если изменить технологические параметры и режимы сушки в сторону уменьшения или увеличения, это приведет к снижению качества готового порошкообразного продукта и повышению энергозатрат на его получение.


