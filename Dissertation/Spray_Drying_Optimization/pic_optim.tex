\begin{figure}[h!]
\center
\begin{small}
\def\svgwidth{13,3cm}
\subfloat[\footnotesize 1~--~2,20; 2~--~2,43; 3~--~2,67; 4~--~2,90; 5~--~3,13;  6~--~3,37; 7~--~3,60; 8~--~3,83; 9~--~4,30]{%
\label{f:sub1}\input{figures/x1x3y1.pdf_tex}}\quad
\end{small}
\caption[]{Кривые равных значений: \subref{f:sub1} удельных энергозатрат, $\text{кВт}\cdot\text{ч}/\text{кг}$, \subref{f:sub2} влагонапряжения сушильной камеры, $\text{кг}/\text{м}^3\cdot\text{ч}$, \subref{f:sub3} влажности готового продукта, $\%$, от температуры сушильного агента $T_\text{с.а.},\text{К}$, и скорости вращения диска сушильной установки $\nu_\text{д},\text{м}/\text{с}$}
\label{x1x3}
\end{figure}



\begin{figure} \ContinuedFloat
\center
\addtocounter{figure}{1}
\begin{small}
\subfloat[\footnotesize 1~--~5,90; 2~--~6,59; 3~--~7,28; 4~--~7,97; 5~--~8,66;  6~--~9,34; 7~--~10,03; 8~--~10,72; 9~--~12,10]{%
\def\svgwidth{13,1cm}
\label{f:sub2}\input{figures/x1x3y2.pdf_tex}}\quad
\end{small}
\begin{small}
\subfloat[\footnotesize 1~--~2,20; 2~--~2,43; 3~--~2,67; 4~--~2,90; 5~--~3,13;  6~--~3,37; 7~--~3,60; 8~--~3,83; 9~--~4,30]{%
\def\svgwidth{13,3cm}
\label{f:sub3}\input{figures/x1x3y3.pdf_tex}}\quad
\end{small}
\caption{Продолжение}
\label{x1x3y1-y2}
\end{figure}


\begin{figure} 
\center
\begin{small}
\def\svgwidth{9.54cm}
\input{figures/x1x3(y1-y3).pdf_tex}
\end{small}
\caption{Номограмма для определения удельных энергозатрат,
$\text{кВт}\cdot\text{ч}/\text{кг}$,  влагонапряжения сушильной камеры, $\text{кг}/\text{м}^3\cdot\text{ч}$,  влажности готового продукта, $\%$}
\end{figure}


\begin{figure}  
\center\begin{small}
\subfloat[][\footnotesize 1~--~2,20; 2~--~2,43; 3~--~2,67; 4~--~2,90; 5~--~3,13;  6~--~3,37; 7~--~3,60; 8~--~3,83; 9~--~4,30]{%
\def\svgwidth{12.5cm}
\label{f:sub4}\input{figures/x2x4y1.pdf_tex}}\quad
\end{small}
\caption[]{Кривые равных значений: \subref{f:sub4} удельных энергозатрат, $\text{кВт}\cdot\text{ч}/\text{кг}$, \subref{f:sub5} влагонапряжения сушильной камеры, $\text{кг}/\text{м}^3\cdot\text{ч}$, \subref{f:sub6} влажности готового продукта, $\%$, от скорости сушильного агента $\nu_\text{с.а.},\text{К}$, и влажности упаренного фильтрата $\omega$~\%}
\label{x1x3}
\end{figure}



\begin{figure} \ContinuedFloat
\center\addtocounter{figure}{1}
\begin{small}
\subfloat[\footnotesize 1~--~5,90; 2~--~6,59; 3~--~7,28; 4~--~7,97; 5~--~8,66;  6~--~9,34; 7~--~10,03; 8~--~10,72; 9~--~12,10]{%
\def\svgwidth{13cm}\label{f:sub5}\input{figures/x2x4y2.pdf_tex}}\quad
\end{small}
\begin{small}
\subfloat[\footnotesize 1~--~2,20; 2~--~2,43; 3~--~2,67; 4~--~2,90; 5~--~3,13;  6~--~3,37; 7~--~3,60; 8~--~3,83; 9~--~4,30]{%
\def\svgwidth{13cm}\label{f:sub6}\input{figures/x2x4y3.pdf_tex}}\quad
\end{small}
\caption{Продолжение}
\label{x1x3y1-y2}
\end{figure}



\begin{figure} 
\center
\begin{small}
\def\svgwidth{11.5cm}
\label{nomo2}\input{figures/x2x4(y1-y3).pdf_tex}
\end{small}
\caption{Номограмма для определения удельных энергозатрат,
$\text{кВт}\cdot\text{ч}/\text{кг}$,  влагонапряжения сушильной камеры, $\text{кг}/\text{м}^3\cdot\text{ч}$,  влажности готового продукта, $\%$}
\end{figure}