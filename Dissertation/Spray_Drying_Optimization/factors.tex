\subsection{Выбор факторов выходной информации}

Критериями оценки влияния входных факторов на процесс сушки послеспиртовой барды были
выбраны:

\begin{description}
\item $Y_1$ -- удельные энергозатраты процесса сушки, отнесенные на 1 кг испаренной влаги,
    {кВт ч}/{кг};
\item $Y_2$ -- влагонапряжение сушильной камеры, кг/(м$\cdot$с);

\item $Y_2$ -- влажность готового продукта, \%
\end{description}

Выбор критериев оценки $Y$ обусловлен их наибольшей значимостью для процесса сушки
послеспиртовой барды.
Так, $Y_1$ определяет энергоемкость процесса и является важным показателем в оценке его
энергетической эффективности, $Y_2$ определяет производительность процесса сушки и напрямую
связан с его скоростью,  $Y_3$ напрямую связан
с качеством готового порошкообразного продукта.
Программа исследования была  заложена в матрицу планирования эксперимента
(табл.~\ref{vhodnpar})


\LTXtable{\textwidth}{tabs/predels}