\subsection{Обоснование интервалов варьировании входных факторов}
Для исследования взаимодействия различных факторов, влияющих на процесс сушки послеспиртовой
барды, были применены
математические методы планирования эксперимента.
Математическое описание данного процесса может быть получено эмпирически. При этом его
математическая модель имеет вид уравнения регрессии, найденного статистическими методами на
основе экспериментов. Математическая модель изучаемого процесса представляется в виде
полинома второй степени: \(N_{j}+^{y}_{h}
\)
\begin{equation}
Y=b_{0}+\sum\limits_{i=1}^{n}b_i\cdot X_i
+\sum\limits_{i=1}^{n}b_{ii}\cdot X_{i}^2
+\sum\limits_{i\leq j}^{n}b_{ij}\cdot X_i\cdot X_j
\end{equation}

\begin{description}
\item где $b_{0}$ -- свободный член уравнения, равный средней величине отклика при
    условии,
    что рассматриваемые факторы находятся на средних, <<нулевых>>, уровнях;
\item $X$ -- масштабированные значения факторов, которые определяют функцию отклика и
    поддаются варьированию;
\item $b_{ij}$ -- коэффициенты двухфакторных взаимодействий, показывающие, насколько
    изменяется степень влияния одного фактора при изменении величины другого;
\item $b_{ii}$ -- коэффициенты  квадратичных эффектов, определяющие нелинейность выходного
    параметра от рассматриваемых факторов;
\item $i$, $j$ -- индексы факторов;
\item $n$ - число факторов в матрице планирования.
\end{description}

В качестве основных факторов, влияющих на процесс сушки послеспиртовой барды, были выбраны:

\begin{description}
\item $X_1$ -- температура сушильного агента, \textordmasculine С;
\item $X_2$ -- скорость сушильного агента, м/с;
\item $X_3$ -- скорость вращения диска распылительной сушилки, м/с
\item $X_4$ -- влажность сгущенного фильтрата барды, \%.
\end{description}
Все эти факторы совместимы и некоррелируемы между собой. Пределы изменения исследуемых
факторов приведены в табл~\ref{vhodparam}.
Выбор интервалов изменения входных факторов обусловлен технологическими условиями процесса
сушки послеспиртовой барды и  технико-экономическими   показателями процесса.
\begin{longtable}{|l|c|c|c|c|c|}
\caption{Пределы изменения входных факторов}
\label{vhodparam}\\
\hline
Условия планирования & Кодирован\-ное  & \multicolumn{4}{c|}{Значение факторов в точках плана}\tabularnewline
\cline{3-6}~ & значение & $X_{1}$ & $X_{2}$ & $X_{3}$ & $X_{4}$\tabularnewline
\cline{3-6}  &  & $T_\text{в}$, \textordmasculine С & $\nu_\text{в}$, м/с & $\nu_\text{д}$, м/с & $\omega$,~\% \\
\hline
Основной уровень & 0 & 70 & 10 & 160 & 65\\
\hline
Интервал планирования & $\Delta$ & 5 & 1 & 20 & 5\\
\hline
Верхний уровень & +1 & 75 & 11 & 180 & 70\\
\hline
Нижний уровень & -1 & 65 & 9 & 140 & 60\\
\hline
Верхняя <<звездная точка>> & +2 & 80 & 12 & 200 & 75\\
\hline
Нижняя <<звездная точка>> & -2 & 60 & 8 & 120 & 55\\
\hline
\end{longtable}


%\LTXtable{\textwidth}{tabs/tab8}
