\subsection{Определение оптимальных интервалов варьирования входных факторов по удельным
энергозатратам сушки,\ влагонапряжение сушильной камеры и  влажности
готового продукта}

Задача оптимизации сформулирована следующим образом, найти такие режимы работы сушки,
которые бы в широком диапазоне изменения входных параметров процесса сушки доставили минимум
удельных энергозатрат, максимум влагонапряжения сушильной камеры и минимум влажности готового продукта.

Общая математическая постановка задачи оптимизации представлена в виде следующей модели:
\begin{equation}
q = q(Y_1, Y_2, Y_3) \to opt~\textrm{при}~x\in D
\end{equation}
Определим область значений:
\begin{eqnarray}
D: Y_1 (X_1, X_2, X_3, X_4)\to min \nonumber\\
Y_2 (X_1, X_2, X_3, X_4)\to max \\
Y_3 (X_1, X_2, X_3, X_4)\to min  \nonumber
\end{eqnarray}
В табл.~\ref{PredelParam} сведены выбранные оптимальные интервалы изменения параметров $X_i$
для всех исследуемых выходных факторов.

\LTXtable{\textwidth}{tabs/parametrs}


Согласно критерию оптимизации  для принятия окончательного решения по выбору оптимальных
режимов исследуемого процесса необходимо решить компромиссную задачу, накладывая
оптимальные, выделенные в табл.~\ref{PredelParam}, интервалы параметров $X_i$ друг на друга.
\clearpage
В результате были получены рациональные значения интервалов входных факторов:
\begin{description}
\item 
$X_1$ = 70\dots77 $\celcius$;
\item 
$X_2$ = 9,5\dots11,5 м/с;
\item 
$X_3$ = 160\dots180 м/с;
\item 
$X_4$ = 58\dots68\%.
\end{description}

Для проверки правильности полученных результатов был поставлен ряд
параллельных экспериментов, полученные результаты попадали в рассчитанные доверительные интервалы по всем критериям качества. При этом среднеквадратичная ошибка не превышала 6\%.
Результаты представлены в табл.~\ref{bardaexperiment3}
