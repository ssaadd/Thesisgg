\subsection{Планирование эксперимента и графическая интерпретация уравнений регрессии}
Дли исследования было применено центральное композиционное ротабельное униформ-планирование,
и был выбран полный факторный эксперимент $2^4$.
При обработке результатов эксперимента были применены следующие статические критерии :
проверка однородности дисперсий -- критерий Кохрена, значимость коэффициентов уравнений
регрессии -- критерий Стьюдента, адекватность уравнений -- критерий Фишера. В результате
статистической обработки экспериментальных  данных получены уравнения регрессии адекватно
описывающие данный процесс, под влиянием исследуемых факторов:

\begin{eqnarray}
\label{regress1}
Y_{1}=3,28+0,11\cdot X_{1}+0,1\cdot X_{2}+0,098\cdot X_{3}+0,067\cdot X_{4}+0,007\cdot
X_{1}\cdot X_{2}+\nonumber \\+0,005\cdot X_{1}\cdot X_{3}+0,0112\cdot X_{1}\cdot
X_{4}+0,0037\cdot X_{2}\cdot X_{3}+0,001\cdot X_{2}\cdot X_{4}+ \\+0,0025\cdot
X_{3}\cdot X_{4}+0,0085\cdot X_{1}^{2}-0,009\cdot X_{2}^{2}-0,0065\cdot
X_{3}^{2}-0,014\cdot
X_{4}^{2}~~~~\nonumber 
\end{eqnarray}
\begin{eqnarray}
\label{regress2}
Y_{2}=9,94-0,23\cdot X_{1}-0,094\cdot X_{2}-0,208\cdot X_{3}+0,204\cdot X_{4}+0,01\cdot
X_{1}\cdot X_{2}+\nonumber \\+0,105\cdot X_{1}\cdot X_{3}+0,0112\cdot X_{1}\cdot
X_{4}+0,0037\cdot X_{2}\cdot X_{3}+0,001\cdot X_{2}\cdot X_{4}-\\-0,025\cdot
X_{3}\cdot X_{4}-0,0185\cdot X_{1}^{2}-0,0048\cdot X_{2}^{2}-0,0035\cdot
X_{3}^{2}-0,002\cdot X_{4}^{2}~~~~\nonumber 
\end{eqnarray}
\begin{eqnarray}
\label{regress3}
Y_{3}=9,13+0,39\cdot X_{1}+0,386\cdot X_{2}+0,387\cdot X_{3}-0,25\cdot X_{4}+0,031\cdot
X_{1}\cdot X_{2}+\nonumber \\+0,032\cdot X_{1}\cdot X_{3}-0,011\cdot X_{1}\cdot
X_{4}+0,034\cdot X_{2}\cdot X_{3}-0,014\cdot X_{2}\cdot X_{4}- \\-0,0143\cdot
X_{3}\cdot X_{4}-0,053\cdot X_{1}^{2}-0,071\cdot X_{2}^{2}-0,055\cdot  X_{3}^{2}-0,091\cdot
X_{4}^{2}~~~~\nonumber 
\end{eqnarray}

Анализ уравнений регрессии (\ref{regress1}) -- (\ref{regress3}) позволяет
выделить факторы, влияющие на рассматриваемый процесс. На критерии оценки наибольшее влияние
оказывают: температура и скорость сушильного агента
на входе в сушильную камеру, наименьшее влажность упаренного фугата после
выпарной установки.
 Причём  знак <<плюс>> перед коэффициентом при линейных членах указывает на то, что при
 увеличении входного параметра изменение  выходного параметра увеличивается.

Степень влияния параметров относительно друг друга в уравнении (\ref{regress1}):
$b_{1}:b_{2}=1,14$; $b_{1}:b_{3}=1,16$; $b_{1}:b_{4}=1,69$; $b_{2}:b_{3}=1,017$;
$b_{2}:b_{4}=1,48$; $b_{3}:b_{4}=1,462$.

Степень влияния параметров относительно друг друга в уравнении (\ref{regress2}):
$b_{1}:b_{2}=1,031$; $b_{1}:b_{3}=1,029$; $b_{1}:b_{4}=1,57$; $b_{2}:b_{3}=0,997$;
$b_{2}:b_{4}=1,521$; $b_{3}:b_{4}=1,525$.

Степень влияния параметров относительно друг друга в уравнении (\ref{regress3}):
$b_{1}:b_{2}=2,49$; $b_{1}:b_{3}=1,124$; $b_{1}:b_{4}=1,143$; $b_{2}:b_{3}=0,45$;
$b_{2}:b_{4}=0,46$; $b_{3}:b_{4}=1,016$.

Полученные уравнения (\ref{regress1}) -- (\ref{regress3}) нелинейные.
В результате выполнения тридцати двух опытов получена информация о влиянии факторов и
построена математическая модель процесса, позволяющая рассчитать удельные энергозатраты,
влагонапряжение объёма сушильной камеры и влажность готового продукта
 внутри выбранных интервалов варьирования входных факторов.
На рис.~\ref{x1x3} показаны кривые равных значений выходных
параметров, которые несут смысл номограмм и представляют
практический интерес.


\begin{figure}[h]
\tiny
\centering
     \begin{subfigure}[b]{0.45\textwidth}
     \def\svgwidth{\textwidth}
     \centering
         \input{figures/3Dy1x1x3.pdf_tex}
         \caption{}
     \end{subfigure}    
     \begin{subfigure}[b]{0.45\textwidth}
     \def\svgwidth{\textwidth}
     \centering
         \input{figures/3Dy2x1x3.pdf_tex}
         \caption{}
     \end{subfigure}
       \begin{subfigure}[b]{0.45\textwidth}
     \def\svgwidth{\textwidth}
     \centering
         \input{figures/3Dy3x1x3.pdf_tex}
         \caption{}
     \end{subfigure}
     \caption{..}
%     \label{fig_parsetree}
\end{figure}


\begin{figure}[htb]
\tiny
\centering
     \begin{subfigure}[b]{0.45\textwidth}
     \def\svgwidth{\textwidth}
     \centering
         \input{figures/3Dy1x2x4.pdf_tex}
         \caption{}
     \end{subfigure}    
     \begin{subfigure}[b]{0.45\textwidth}
     \def\svgwidth{\textwidth}
     \centering
         \input{figures/3Dy2x2x4.pdf_tex}
         \caption{}
     \end{subfigure}
       \begin{subfigure}[b]{0.45\textwidth}
     \def\svgwidth{\textwidth}
     \centering
         \input{figures/3Dy3x2x4.pdf_tex}
         \caption{}
     \end{subfigure}
     \caption{..}
%     \label{fig_parsetree}
\end{figure}


\begin{figure}[htb]
\tiny
\centering
     \begin{subfigure}[b]{0.453\textwidth}
     \def\svgwidth{\textwidth}
     \centering
         \input{figures/x1x3nomogramme.pdf_tex}
         \caption{}
     \end{subfigure}    
     \begin{subfigure}[b]{0.45\textwidth}
     \def\svgwidth{\textwidth}
     \centering
         \input{figures/x2x4nomogramme.pdf_tex}
         \caption{}
     \end{subfigure}
     \caption{Номограммы}
%     \label{}
\end{figure}

%\begin{figure}[h!]
\center
\begin{small}
\def\svgwidth{13,3cm}
\subfloat[\footnotesize 1~--~2,20; 2~--~2,43; 3~--~2,67; 4~--~2,90; 5~--~3,13;  6~--~3,37; 7~--~3,60; 8~--~3,83; 9~--~4,30]{%
\label{f:sub1}\input{figures/x1x3y1.pdf_tex}}\quad
\end{small}
\caption[]{Кривые равных значений: \subref{f:sub1} удельных энергозатрат, $\text{кВт}\cdot\text{ч}/\text{кг}$, \subref{f:sub2} влагонапряжения сушильной камеры, $\text{кг}/\text{м}^3\cdot\text{ч}$, \subref{f:sub3} влажности готового продукта, $\%$, от температуры сушильного агента $T_\text{с.а.},\text{К}$, и скорости вращения диска сушильной установки $\nu_\text{д},\text{м}/\text{с}$}
\label{x1x3}
\end{figure}



\begin{figure} \ContinuedFloat
\center
\addtocounter{figure}{1}
\begin{small}
\subfloat[\footnotesize 1~--~5,90; 2~--~6,59; 3~--~7,28; 4~--~7,97; 5~--~8,66;  6~--~9,34; 7~--~10,03; 8~--~10,72; 9~--~12,10]{%
\def\svgwidth{13,1cm}
\label{f:sub2}\input{figures/x1x3y2.pdf_tex}}\quad
\end{small}
\begin{small}
\subfloat[\footnotesize 1~--~2,20; 2~--~2,43; 3~--~2,67; 4~--~2,90; 5~--~3,13;  6~--~3,37; 7~--~3,60; 8~--~3,83; 9~--~4,30]{%
\def\svgwidth{13,3cm}
\label{f:sub3}\input{figures/x1x3y3.pdf_tex}}\quad
\end{small}
\caption{Продолжение}
\label{x1x3y1-y2}
\end{figure}


\begin{figure} 
\center
\begin{small}
\def\svgwidth{9.54cm}
\input{figures/x1x3(y1-y3).pdf_tex}
\end{small}
\caption{Номограмма для определения удельных энергозатрат,
$\text{кВт}\cdot\text{ч}/\text{кг}$,  влагонапряжения сушильной камеры, $\text{кг}/\text{м}^3\cdot\text{ч}$,  влажности готового продукта, $\%$}
\end{figure}


\begin{figure}  
\center\begin{small}
\subfloat[][\footnotesize 1~--~2,20; 2~--~2,43; 3~--~2,67; 4~--~2,90; 5~--~3,13;  6~--~3,37; 7~--~3,60; 8~--~3,83; 9~--~4,30]{%
\def\svgwidth{12.5cm}
\label{f:sub4}\input{figures/x2x4y1.pdf_tex}}\quad
\end{small}
\caption[]{Кривые равных значений: \subref{f:sub4} удельных энергозатрат, $\text{кВт}\cdot\text{ч}/\text{кг}$, \subref{f:sub5} влагонапряжения сушильной камеры, $\text{кг}/\text{м}^3\cdot\text{ч}$, \subref{f:sub6} влажности готового продукта, $\%$, от скорости сушильного агента $\nu_\text{с.а.},\text{К}$, и влажности упаренного фильтрата $\omega$~\%}
\label{x1x3}
\end{figure}



\begin{figure} \ContinuedFloat
\center\addtocounter{figure}{1}
\begin{small}
\subfloat[\footnotesize 1~--~5,90; 2~--~6,59; 3~--~7,28; 4~--~7,97; 5~--~8,66;  6~--~9,34; 7~--~10,03; 8~--~10,72; 9~--~12,10]{%
\def\svgwidth{13cm}\label{f:sub5}\input{figures/x2x4y2.pdf_tex}}\quad
\end{small}
\begin{small}
\subfloat[\footnotesize 1~--~2,20; 2~--~2,43; 3~--~2,67; 4~--~2,90; 5~--~3,13;  6~--~3,37; 7~--~3,60; 8~--~3,83; 9~--~4,30]{%
\def\svgwidth{13cm}\label{f:sub6}\input{figures/x2x4y3.pdf_tex}}\quad
\end{small}
\caption{Продолжение}
\label{x1x3y1-y2}
\end{figure}



\begin{figure} 
\center
\begin{small}
\def\svgwidth{11.5cm}
\label{nomo2}\input{figures/x2x4(y1-y3).pdf_tex}
\end{small}
\caption{Номограмма для определения удельных энергозатрат,
$\text{кВт}\cdot\text{ч}/\text{кг}$,  влагонапряжения сушильной камеры, $\text{кг}/\text{м}^3\cdot\text{ч}$,  влажности готового продукта, $\%$}
\end{figure}
