\begin{longtable}{|X|Y|Y|Y|Y|}
\caption{\label{tab:fraction_sostav}Фракционный состав белка барды}
\tabularnewline
\hline
\multicolumn{1}{|c|}{\multirow{2}[4]{*}{Показатель}} & \multicolumn{4}{c|}{Барда} \\
\cline{2-5}\multicolumn{1}{|c|}{} & пшеничная & кукурузная & ячменная & просяная \\
\hline
\multicolumn{4}{|l|}{Белки и аминокислоты, \%:}  \\
\hline
\multicolumn{1}{|l|}{метод Къельдаля (по общему азоту)} & 46,4\(\pm\)2,3 ж; 2,5\(\pm\)0,1 т & 36,0\(\pm\)1,8 ж; 2,5\(\pm\)0,1 т & 27,5\(\pm\)1,3 ж; 2,2\(\pm\)0,1 т & 39,1\(\pm\)1,9 ж; 2,1\(\pm\)0,1 т \\
\hline
\multicolumn{1}{|l|}{СФ по реакции с нингидрином} & 35,7\(\pm\)1,7 ж & 28,8\(\pm\)1,4 ж & 21,7\(\pm\)1,1 ж & 32,\(\pm\)1,6 ж \\
\hline
\multicolumn{1}{|l|}{ВЭЖХ} & 30,9\(\pm\)1,5 ж & 28,9\(\pm\)1,5 ж & 20,4\(\pm\)1,0 ж & 32,4\(\pm\)1,6 ж \\
\hline
\multicolumn{1}{|l|}{в том числе незаменимые аминок-ты} & 12,4\(\pm\)0,4 ж & 14,2\(\pm\)0,4 ж & 10,1\(\pm\)0,3 ж & 13,2\(\pm\)0,4 ж \\
\hline
\multicolumn{1}{|l|}{Высоко- и среднемолекулярные, \%} & 17,2,:67,7:15,1 & 25,0:59,1:15,9 & 57,9:31,6:10,5 & 34,1:47,7:18,2 \\
\hline
\multicolumn{1}{|l|}{Альбумины, глобулины, проламины, \%} & 16,7:7,4:40,7:29,6 & 15,7:6,5:40,7:30,6 & 13,2:5,1:35,5:39,3 & 15,9:8,5:38,8:29,7 \\
\hline
\multicolumn{1}{|l|}{Пентозы (гравиметрия), \%} & 3,5\(\pm\)0,2 ж; 0,2\(\pm\)0,02 т & 4,6\(\pm\)0,2 ж; 0,50\(\pm\)0,03 т & 4,2\(\pm\)0,2 ж; 0,9\(\pm\)0,03 т & 4,5\(\pm\)0,2 ж; 0,61\(\pm\)0,03 т \\
\hline
\multicolumn{1}{|l|}{Пентозаны (гравиметрия), \%} & 3,1\(\pm\)0,2 ж; 0,35\(\pm\)0,02 т & 4,1\(\pm\)0,2 ж; 0,44\(\pm\)0,02 т & 3,7\(\pm\)0,2 ж; 0,44\(\pm\)0,02 т & 4,4\(\pm\)0,2 ж; 0,54\(\pm\)0,02 т \\
\hline
\multicolumn{1}{|l|}{Целлюлоза (гравиметрия), \%} & 0,7\(\pm\)0,0З т & 0,5\(\pm\)0,02 т & 0,7\(\pm\)0,3 т & 0,9\(\pm\)0,04 т \\
\hline
\multicolumn{1}{|l|}{Жирное масло (гравиметрия), \%} & 8,4\(\pm\)0,4 & 11,1\(\pm\)0,5 & 8,4\(\pm\)0,4 & 10,4\(\pm\)0,5 \\
\hline
\multicolumn{1}{|l|}{Аскорбиновая кислота, мг \%} & 6,2\(\pm\)0,3 & 11,4\(\pm\)0,5 & --     & -- \\
\hline
\multicolumn{1}{|l|}{Токоферолы (УФ СФ), мг \%} & 3,4\(\pm\)0,2 & 7,7\(\pm\)0,4 & 4,4\(\pm\)0,2 & -- \\
\hline
\multicolumn{1}{|l|}{Флавоноиды (СФ), \%} & 0,89\(\pm\)0,04 & 0,41\(\pm\)0,02 & 0,78\(\pm\)0,4 & 0,35\(\pm\)0,2 \\
\hline
\multicolumn{5}{|l|}{* ж, т "--- жидкая и твердая фаза соответственно} \\
\hline
\end{longtable}%
