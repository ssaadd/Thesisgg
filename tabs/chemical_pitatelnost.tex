\begin{longtable}{|X|Y|Y|Y|Y|}
\caption{\label{tab:Pitatel}Химический состав и питательность 100г послеспиртовой барды разного происхождения,
\% в.с.в.}
\tabularnewline
\hline
\multirow{2}{*}{Показатели} & \multicolumn{3}{c|}{Данные ВНИТИП}\tabularnewline
\cline{2-4}
 & \multirow{1}{*}{Ячменная} & \multirow{1}{*}{Кукурузная} & \multirow{1}{*}{Ячменная и}\tabularnewline
 &  &  &зерновая в среднем \tabularnewline
  \hline
\centering{\textbf{1}} & \textbf{2} & \textbf{3} & \textbf{4} \\
\endfirsthead
\caption*{Продолжение таблицы \ref{tab:Pitatel}}
\\
\hline
\centering{\textbf{1}} & \textbf{2} & \textbf{3} & \textbf{4} \\
\endhead
\hline
Влажность & 11,0 & 10,7 & 7,6\tabularnewline
\hline
Сухое вещество & 89,0 & 89,3 & 92,4\tabularnewline
\hline
Обменная энергия & 215 & 240 & 215\\
\hline
Сырой протеин & 24,5 & 27,1 & 26,1\\
\hline
Сырой жир & 5,3 & 12,4 & 5,1\\
\hline
Сырая клетчатка & 12,1 & 5,3 & 15,1\\
\hline
Сырая зола & 4,5 & - & -\\
\hline
Кальций & 0,30 & 0,11 & 0,13\\
\hline
Фосфор общий & 0,53 & 0,78 & 0,35\\
\hline
Натрий & 0,06 & 0,21 & 0,10\\
\hline
Лизин & 0,47 & 0,80 & 0,85\\
\hline
Метионин & 0,55 & 0,87 & 0,65\\
\hline
Цистеин & 0,41 & 0,47 & 0,38\\
\hline
Аргинин & 1,22 & 1,50 & 1,16\\
\hline
Гистидин & - & 0,87 & 0,83\\
\hline
Лейцин & - & 3,00 & 2,88\\
\hline
Изолейцин & - & 0,99 & 1,03\\
\hline
Фенилаланин & - & - & 1,24\\
\hline
Тирозин & - & - & 0,87\\
\hline
Треонин & 0,90 & - & 1,00\\
\hline
Валин & - & - & 1,28\\
\hline
Глицин & - & - & 0,99\\
\hline
\end{longtable}

