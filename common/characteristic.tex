{\actuality} Обзор, введение в тему, обозначение места данной работы в
мировых исследованиях и~т.\:п.

 \aim\ данной работы является разработка научно обоснованной ресурсосберегающей, экологически безопасной и энергоэффективной технологии утилизации фильтрата спиртовой барды при получении кормовых средств; разработка рекомендаций по проектированию и внедрению в производство высокоэффективных конструкций баромембранных и барботажных выпарных аппаратов для минимизации удельных теплоэнергетические потерь и повышения качества готового продукта.

Для~достижения поставленной цели необходимо было решить следующие {\tasks}:

-- Исследование теплофизических свойств фильтрата спиртовой барды как объекта концентрирования при получении сухого порошка.
%-- Изучение порошкообразного продукта из фильтрата спиртовой барды как объекта кормопроизводства;

-- Разработка научно--практических подходов к энергосбережению процессов барботажного выпаривания и распылительной сушки  с компромиссом между качеством готового продукта и удельными энергетическими затратами.

-- Выполнение комплексных экспериментальных и теоретических исследований кинетических закономерностей процесса барботажного выпаривания фильтрата спиртовой барды горячим воздухом.

--  Разработка математической модели процесса барботажного выпаривания фильтрата спиртовой барды на основе законов Фурье и Фика. Предложение метода численно--аналитического решения задачи определения массового расхода пара из фильтрата барды через поверхность пузырька воздуха и выполнение идентификации параметров модели по экспериментальным данным.

-- Разработка конструкций барботажного выпарного и мембранных аппаратов для реализации процессов концентрирования фильтрата барды.

-- Разработка научно--практических подходов к созданию энергоэффективной технологии получения порошка из фильтрата барды с использованием холодильной техники.
%-- Разработка конструкции мембранного аппарата, обеспечивающего повышение эффективности мембранного разделения при концентрировании фильтрата барды за счет снижения поляризационной концентрации.

-- Разработка программно--логического алгоритма управления технологическими параметрами при получении порошка из фильтрата барды, обеспечивающего наименьшие потери теплоты и электроэнергии.

-- Промышленная апробация, технико--экономическая оценка и эксергетический анализ предлагаемой технологии получения порошка из фильтрата барды.


\defpositions

-- результаты экспериментальных исследований кинетических закономерностей процесса барботажного выпаривания фильтрата спиртовой
барды горячим воздухом;

-- математическая модель процесса барботажного  выпаривания фильтрата барды;

-- предлагаемый способ получения порошка из фильтрата барды с использованием рекуперации и утилизации вторичных энергоресурсов;

-- программно--логический алгоритм управления технологическими параметрами при получении порошка из фильтрата барды.

\novelty

-- Выявлены закономерности кинетики барботажного выпаривания фильтрата барды; получены уравнения кинетики массового потока пара; определены численные значения и диапазон изменения основных кинетических характеристик.
-- Предложено численно--аналитическое решение задачи определения массового расхода пара из фильтрата барды через поверхность пузырька воздуха на
основе законов Фурье и Фика.

-- Составлен программно--логический алгоритм управления технологическими параметрами процесса получения порошка из фильтрата барды на базе пароэжекторного теплового насоса, обеспечивающий повышение энергетической эффективности совместно протекающих процессов выпаривания и распылительной сушки.

-- Выполнен эксергетический анализ и проведена оценка термодинамического совершенства способа получения порошка из фильтрата барды как системы процессов.

-- Научная новизна предложенных технических решений подтверждена 5 патентами РФ и 1 свидетельством РОСПАТЕНТА о государственной регистрации программы для ЭВМ.


\influence\ Результаты, изложенные в диссертации, могут быть использованы для ...


\reliability\ Разработаны программа для ЭВМ (свид. Роспатента о гос. регистрации №~2015619721) и программно--логический алгоритм  системы оптимального управления процессами распылительной сушки и переработки фильтрата барды на порошок (Пат. РФ №~2546214), позволяющие эффективно использовать отработанные теплоносители и обеспечить снижение удельных энергозатрат на 10\ldots15~\%.

Выполнен эксергетический анализ процесса  получения порошка из фильтрата спиртовой барды, свидетельствующий о термодинамическом совершенстве предлагаемого
способа утилизации фильтрата барды.

Научная новизна предложенных технических решений отражена в 5 патентах РФ на изобретения.
Продана лицензия (договор №~26/15 от 26.10.2015~г.) на право использования интеллектуальной собственности предприятию ООО <<Пивное ремесло>>, по патенту на изобретение РФ №~2558894.
Годовой экономический эффект от внедрения предлагаемых технических решений составит \(0,9\) млн. р.

%Достоверность научных разработок подтверждена промышленными испытаниями предлагаемых способов сушки пищевого растительного сырья в СПК «Воронежский тепличный комбинат» и в ООО «СуперАгро», а также актом внедрения сушильной установки по патенту РФ № 2520752 в рамках реализации программы Союзного государства России и Республики Беларусь «Разработка перспективных ресурсосберегающих, экологически чистых технологий и оборудования для производства биологически полноценных комбикормов» на 2011–2013 годы (ОАО «ВНИИКП»).

Разработана энергоэффективная технология получения порошкообразного продукта из фильтрата спиртовой барды (Пат. РФ №~2514666).

Разработаны конструкции баромембранных аппаратов (Пат. РФ №~2560417, №~2558894).

Проведены производственные испытания в условиях ОАО <<>>, которые показали высокую эффективность предлагаемых технических и технологических решений.

\probation\
Основные результаты работы докладывались~на:
перечисление основных конференций, симпозиумов и~т.\:п.

\contribution\ Содержание диссертации и основные положения, выносимые на защиту, отражают персональный вклад автора в опубликованные работы.
Подготовка к публикации полученных результатов проводилась совместно с соавторами, причем вклад диссертанта был определяющим. Все представленные в диссертации результаты получены лично автором.


%\publications\ Основные результаты по теме диссертации изложены в ХХ печатных изданиях~\cite{Sokolov,Gaidaenko,Lermontov,Management},
%Х из которых изданы в журналах, рекомендованных ВАК~\cite{Sokolov,Gaidaenko}, 
%ХХ --- в тезисах докладов~\cite{Lermontov,Management}.

\ifthenelse{\equal{\thebibliosel}{0}}{% Встроенная реализация с загрузкой файла через движок bibtex8
    \publications\ Основные результаты по теме диссертации изложены в XX печатных изданиях, 
    X из которых изданы в журналах, рекомендованных ВАК, 
    X "--- в тезисах докладов.%
}{% Реализация пакетом biblatex через движок biber
%Сделана отдельная секция, чтобы не отображались в списке цитированных материалов
    \begin{refsection}%
        \printbibliography[heading=countauthornotvak, env=countauthornotvak, keyword=biblioauthornotvak, section=1]%
        \printbibliography[heading=countauthorvak, env=countauthorvak, keyword=biblioauthorvak, section=1]%
        \printbibliography[heading=countauthorconf, env=countauthorconf, keyword=biblioauthorconf, section=1]%
        \printbibliography[heading=countauthor, env=countauthor, keyword=biblioauthor, section=1]%
        \publications\ Основные результаты по теме диссертации изложены в \arabic{citeauthor} печатных изданиях\nocite{bib1,bib2}, 
        \arabic{citeauthorvak} из которых изданы в журналах, рекомендованных ВАК\nocite{vakbib1,vakbib2}, 
        \arabic{citeauthorconf} "--- в тезисах докладов\nocite{confbib1,confbib2}.
    \end{refsection}
}